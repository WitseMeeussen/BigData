\documentclass[11pt]{article}

    \usepackage[breakable]{tcolorbox}
    \usepackage{parskip} % Stop auto-indenting (to mimic markdown behaviour)
    
    \usepackage{iftex}
    \ifPDFTeX
    	\usepackage[T1]{fontenc}
    	\usepackage{mathpazo}
    \else
    	\usepackage{fontspec}
    \fi

    % Basic figure setup, for now with no caption control since it's done
    % automatically by Pandoc (which extracts ![](path) syntax from Markdown).
    \usepackage{graphicx}
    % Maintain compatibility with old templates. Remove in nbconvert 6.0
    \let\Oldincludegraphics\includegraphics
    % Ensure that by default, figures have no caption (until we provide a
    % proper Figure object with a Caption API and a way to capture that
    % in the conversion process - todo).
    \usepackage{caption}
    \DeclareCaptionFormat{nocaption}{}
    \captionsetup{format=nocaption,aboveskip=0pt,belowskip=0pt}

    \usepackage{float}
    \floatplacement{figure}{H} % forces figures to be placed at the correct location
    \usepackage{xcolor} % Allow colors to be defined
    \usepackage{enumerate} % Needed for markdown enumerations to work
    \usepackage{geometry} % Used to adjust the document margins
    \usepackage{amsmath} % Equations
    \usepackage{amssymb} % Equations
    \usepackage{textcomp} % defines textquotesingle
    % Hack from http://tex.stackexchange.com/a/47451/13684:
    \AtBeginDocument{%
        \def\PYZsq{\textquotesingle}% Upright quotes in Pygmentized code
    }
    \usepackage{upquote} % Upright quotes for verbatim code
    \usepackage{eurosym} % defines \euro
    \usepackage[mathletters]{ucs} % Extended unicode (utf-8) support
    \usepackage{fancyvrb} % verbatim replacement that allows latex
    \usepackage{grffile} % extends the file name processing of package graphics 
                         % to support a larger range
    \makeatletter % fix for old versions of grffile with XeLaTeX
    \@ifpackagelater{grffile}{2019/11/01}
    {
      % Do nothing on new versions
    }
    {
      \def\Gread@@xetex#1{%
        \IfFileExists{"\Gin@base".bb}%
        {\Gread@eps{\Gin@base.bb}}%
        {\Gread@@xetex@aux#1}%
      }
    }
    \makeatother
    \usepackage[Export]{adjustbox} % Used to constrain images to a maximum size
    \adjustboxset{max size={0.9\linewidth}{0.9\paperheight}}

    % The hyperref package gives us a pdf with properly built
    % internal navigation ('pdf bookmarks' for the table of contents,
    % internal cross-reference links, web links for URLs, etc.)
    \usepackage{hyperref}
    % The default LaTeX title has an obnoxious amount of whitespace. By default,
    % titling removes some of it. It also provides customization options.
    \usepackage{titling}
    \usepackage{longtable} % longtable support required by pandoc >1.10
    \usepackage{booktabs}  % table support for pandoc > 1.12.2
    \usepackage[inline]{enumitem} % IRkernel/repr support (it uses the enumerate* environment)
    \usepackage[normalem]{ulem} % ulem is needed to support strikethroughs (\sout)
                                % normalem makes italics be italics, not underlines
    \usepackage{mathrsfs}
    

    
    % Colors for the hyperref package
    \definecolor{urlcolor}{rgb}{0,.145,.698}
    \definecolor{linkcolor}{rgb}{.71,0.21,0.01}
    \definecolor{citecolor}{rgb}{.12,.54,.11}

    % ANSI colors
    \definecolor{ansi-black}{HTML}{3E424D}
    \definecolor{ansi-black-intense}{HTML}{282C36}
    \definecolor{ansi-red}{HTML}{E75C58}
    \definecolor{ansi-red-intense}{HTML}{B22B31}
    \definecolor{ansi-green}{HTML}{00A250}
    \definecolor{ansi-green-intense}{HTML}{007427}
    \definecolor{ansi-yellow}{HTML}{DDB62B}
    \definecolor{ansi-yellow-intense}{HTML}{B27D12}
    \definecolor{ansi-blue}{HTML}{208FFB}
    \definecolor{ansi-blue-intense}{HTML}{0065CA}
    \definecolor{ansi-magenta}{HTML}{D160C4}
    \definecolor{ansi-magenta-intense}{HTML}{A03196}
    \definecolor{ansi-cyan}{HTML}{60C6C8}
    \definecolor{ansi-cyan-intense}{HTML}{258F8F}
    \definecolor{ansi-white}{HTML}{C5C1B4}
    \definecolor{ansi-white-intense}{HTML}{A1A6B2}
    \definecolor{ansi-default-inverse-fg}{HTML}{FFFFFF}
    \definecolor{ansi-default-inverse-bg}{HTML}{000000}

    % common color for the border for error outputs.
    \definecolor{outerrorbackground}{HTML}{FFDFDF}

    % commands and environments needed by pandoc snippets
    % extracted from the output of `pandoc -s`
    \providecommand{\tightlist}{%
      \setlength{\itemsep}{0pt}\setlength{\parskip}{0pt}}
    \DefineVerbatimEnvironment{Highlighting}{Verbatim}{commandchars=\\\{\}}
    % Add ',fontsize=\small' for more characters per line
    \newenvironment{Shaded}{}{}
    \newcommand{\KeywordTok}[1]{\textcolor[rgb]{0.00,0.44,0.13}{\textbf{{#1}}}}
    \newcommand{\DataTypeTok}[1]{\textcolor[rgb]{0.56,0.13,0.00}{{#1}}}
    \newcommand{\DecValTok}[1]{\textcolor[rgb]{0.25,0.63,0.44}{{#1}}}
    \newcommand{\BaseNTok}[1]{\textcolor[rgb]{0.25,0.63,0.44}{{#1}}}
    \newcommand{\FloatTok}[1]{\textcolor[rgb]{0.25,0.63,0.44}{{#1}}}
    \newcommand{\CharTok}[1]{\textcolor[rgb]{0.25,0.44,0.63}{{#1}}}
    \newcommand{\StringTok}[1]{\textcolor[rgb]{0.25,0.44,0.63}{{#1}}}
    \newcommand{\CommentTok}[1]{\textcolor[rgb]{0.38,0.63,0.69}{\textit{{#1}}}}
    \newcommand{\OtherTok}[1]{\textcolor[rgb]{0.00,0.44,0.13}{{#1}}}
    \newcommand{\AlertTok}[1]{\textcolor[rgb]{1.00,0.00,0.00}{\textbf{{#1}}}}
    \newcommand{\FunctionTok}[1]{\textcolor[rgb]{0.02,0.16,0.49}{{#1}}}
    \newcommand{\RegionMarkerTok}[1]{{#1}}
    \newcommand{\ErrorTok}[1]{\textcolor[rgb]{1.00,0.00,0.00}{\textbf{{#1}}}}
    \newcommand{\NormalTok}[1]{{#1}}
    
    % Additional commands for more recent versions of Pandoc
    \newcommand{\ConstantTok}[1]{\textcolor[rgb]{0.53,0.00,0.00}{{#1}}}
    \newcommand{\SpecialCharTok}[1]{\textcolor[rgb]{0.25,0.44,0.63}{{#1}}}
    \newcommand{\VerbatimStringTok}[1]{\textcolor[rgb]{0.25,0.44,0.63}{{#1}}}
    \newcommand{\SpecialStringTok}[1]{\textcolor[rgb]{0.73,0.40,0.53}{{#1}}}
    \newcommand{\ImportTok}[1]{{#1}}
    \newcommand{\DocumentationTok}[1]{\textcolor[rgb]{0.73,0.13,0.13}{\textit{{#1}}}}
    \newcommand{\AnnotationTok}[1]{\textcolor[rgb]{0.38,0.63,0.69}{\textbf{\textit{{#1}}}}}
    \newcommand{\CommentVarTok}[1]{\textcolor[rgb]{0.38,0.63,0.69}{\textbf{\textit{{#1}}}}}
    \newcommand{\VariableTok}[1]{\textcolor[rgb]{0.10,0.09,0.49}{{#1}}}
    \newcommand{\ControlFlowTok}[1]{\textcolor[rgb]{0.00,0.44,0.13}{\textbf{{#1}}}}
    \newcommand{\OperatorTok}[1]{\textcolor[rgb]{0.40,0.40,0.40}{{#1}}}
    \newcommand{\BuiltInTok}[1]{{#1}}
    \newcommand{\ExtensionTok}[1]{{#1}}
    \newcommand{\PreprocessorTok}[1]{\textcolor[rgb]{0.74,0.48,0.00}{{#1}}}
    \newcommand{\AttributeTok}[1]{\textcolor[rgb]{0.49,0.56,0.16}{{#1}}}
    \newcommand{\InformationTok}[1]{\textcolor[rgb]{0.38,0.63,0.69}{\textbf{\textit{{#1}}}}}
    \newcommand{\WarningTok}[1]{\textcolor[rgb]{0.38,0.63,0.69}{\textbf{\textit{{#1}}}}}
    
    
    % Define a nice break command that doesn't care if a line doesn't already
    % exist.
    \def\br{\hspace*{\fill} \\* }
    % Math Jax compatibility definitions
    \def\gt{>}
    \def\lt{<}
    \let\Oldtex\TeX
    \let\Oldlatex\LaTeX
    \renewcommand{\TeX}{\textrm{\Oldtex}}
    \renewcommand{\LaTeX}{\textrm{\Oldlatex}}
    % Document parameters
    % Document title
    \title{BigData: Architectural Categorization}
    
    
    
    
    
% Pygments definitions
\makeatletter
\def\PY@reset{\let\PY@it=\relax \let\PY@bf=\relax%
    \let\PY@ul=\relax \let\PY@tc=\relax%
    \let\PY@bc=\relax \let\PY@ff=\relax}
\def\PY@tok#1{\csname PY@tok@#1\endcsname}
\def\PY@toks#1+{\ifx\relax#1\empty\else%
    \PY@tok{#1}\expandafter\PY@toks\fi}
\def\PY@do#1{\PY@bc{\PY@tc{\PY@ul{%
    \PY@it{\PY@bf{\PY@ff{#1}}}}}}}
\def\PY#1#2{\PY@reset\PY@toks#1+\relax+\PY@do{#2}}

\@namedef{PY@tok@w}{\def\PY@tc##1{\textcolor[rgb]{0.73,0.73,0.73}{##1}}}
\@namedef{PY@tok@c}{\let\PY@it=\textit\def\PY@tc##1{\textcolor[rgb]{0.25,0.50,0.50}{##1}}}
\@namedef{PY@tok@cp}{\def\PY@tc##1{\textcolor[rgb]{0.74,0.48,0.00}{##1}}}
\@namedef{PY@tok@k}{\let\PY@bf=\textbf\def\PY@tc##1{\textcolor[rgb]{0.00,0.50,0.00}{##1}}}
\@namedef{PY@tok@kp}{\def\PY@tc##1{\textcolor[rgb]{0.00,0.50,0.00}{##1}}}
\@namedef{PY@tok@kt}{\def\PY@tc##1{\textcolor[rgb]{0.69,0.00,0.25}{##1}}}
\@namedef{PY@tok@o}{\def\PY@tc##1{\textcolor[rgb]{0.40,0.40,0.40}{##1}}}
\@namedef{PY@tok@ow}{\let\PY@bf=\textbf\def\PY@tc##1{\textcolor[rgb]{0.67,0.13,1.00}{##1}}}
\@namedef{PY@tok@nb}{\def\PY@tc##1{\textcolor[rgb]{0.00,0.50,0.00}{##1}}}
\@namedef{PY@tok@nf}{\def\PY@tc##1{\textcolor[rgb]{0.00,0.00,1.00}{##1}}}
\@namedef{PY@tok@nc}{\let\PY@bf=\textbf\def\PY@tc##1{\textcolor[rgb]{0.00,0.00,1.00}{##1}}}
\@namedef{PY@tok@nn}{\let\PY@bf=\textbf\def\PY@tc##1{\textcolor[rgb]{0.00,0.00,1.00}{##1}}}
\@namedef{PY@tok@ne}{\let\PY@bf=\textbf\def\PY@tc##1{\textcolor[rgb]{0.82,0.25,0.23}{##1}}}
\@namedef{PY@tok@nv}{\def\PY@tc##1{\textcolor[rgb]{0.10,0.09,0.49}{##1}}}
\@namedef{PY@tok@no}{\def\PY@tc##1{\textcolor[rgb]{0.53,0.00,0.00}{##1}}}
\@namedef{PY@tok@nl}{\def\PY@tc##1{\textcolor[rgb]{0.63,0.63,0.00}{##1}}}
\@namedef{PY@tok@ni}{\let\PY@bf=\textbf\def\PY@tc##1{\textcolor[rgb]{0.60,0.60,0.60}{##1}}}
\@namedef{PY@tok@na}{\def\PY@tc##1{\textcolor[rgb]{0.49,0.56,0.16}{##1}}}
\@namedef{PY@tok@nt}{\let\PY@bf=\textbf\def\PY@tc##1{\textcolor[rgb]{0.00,0.50,0.00}{##1}}}
\@namedef{PY@tok@nd}{\def\PY@tc##1{\textcolor[rgb]{0.67,0.13,1.00}{##1}}}
\@namedef{PY@tok@s}{\def\PY@tc##1{\textcolor[rgb]{0.73,0.13,0.13}{##1}}}
\@namedef{PY@tok@sd}{\let\PY@it=\textit\def\PY@tc##1{\textcolor[rgb]{0.73,0.13,0.13}{##1}}}
\@namedef{PY@tok@si}{\let\PY@bf=\textbf\def\PY@tc##1{\textcolor[rgb]{0.73,0.40,0.53}{##1}}}
\@namedef{PY@tok@se}{\let\PY@bf=\textbf\def\PY@tc##1{\textcolor[rgb]{0.73,0.40,0.13}{##1}}}
\@namedef{PY@tok@sr}{\def\PY@tc##1{\textcolor[rgb]{0.73,0.40,0.53}{##1}}}
\@namedef{PY@tok@ss}{\def\PY@tc##1{\textcolor[rgb]{0.10,0.09,0.49}{##1}}}
\@namedef{PY@tok@sx}{\def\PY@tc##1{\textcolor[rgb]{0.00,0.50,0.00}{##1}}}
\@namedef{PY@tok@m}{\def\PY@tc##1{\textcolor[rgb]{0.40,0.40,0.40}{##1}}}
\@namedef{PY@tok@gh}{\let\PY@bf=\textbf\def\PY@tc##1{\textcolor[rgb]{0.00,0.00,0.50}{##1}}}
\@namedef{PY@tok@gu}{\let\PY@bf=\textbf\def\PY@tc##1{\textcolor[rgb]{0.50,0.00,0.50}{##1}}}
\@namedef{PY@tok@gd}{\def\PY@tc##1{\textcolor[rgb]{0.63,0.00,0.00}{##1}}}
\@namedef{PY@tok@gi}{\def\PY@tc##1{\textcolor[rgb]{0.00,0.63,0.00}{##1}}}
\@namedef{PY@tok@gr}{\def\PY@tc##1{\textcolor[rgb]{1.00,0.00,0.00}{##1}}}
\@namedef{PY@tok@ge}{\let\PY@it=\textit}
\@namedef{PY@tok@gs}{\let\PY@bf=\textbf}
\@namedef{PY@tok@gp}{\let\PY@bf=\textbf\def\PY@tc##1{\textcolor[rgb]{0.00,0.00,0.50}{##1}}}
\@namedef{PY@tok@go}{\def\PY@tc##1{\textcolor[rgb]{0.53,0.53,0.53}{##1}}}
\@namedef{PY@tok@gt}{\def\PY@tc##1{\textcolor[rgb]{0.00,0.27,0.87}{##1}}}
\@namedef{PY@tok@err}{\def\PY@bc##1{{\setlength{\fboxsep}{\string -\fboxrule}\fcolorbox[rgb]{1.00,0.00,0.00}{1,1,1}{\strut ##1}}}}
\@namedef{PY@tok@kc}{\let\PY@bf=\textbf\def\PY@tc##1{\textcolor[rgb]{0.00,0.50,0.00}{##1}}}
\@namedef{PY@tok@kd}{\let\PY@bf=\textbf\def\PY@tc##1{\textcolor[rgb]{0.00,0.50,0.00}{##1}}}
\@namedef{PY@tok@kn}{\let\PY@bf=\textbf\def\PY@tc##1{\textcolor[rgb]{0.00,0.50,0.00}{##1}}}
\@namedef{PY@tok@kr}{\let\PY@bf=\textbf\def\PY@tc##1{\textcolor[rgb]{0.00,0.50,0.00}{##1}}}
\@namedef{PY@tok@bp}{\def\PY@tc##1{\textcolor[rgb]{0.00,0.50,0.00}{##1}}}
\@namedef{PY@tok@fm}{\def\PY@tc##1{\textcolor[rgb]{0.00,0.00,1.00}{##1}}}
\@namedef{PY@tok@vc}{\def\PY@tc##1{\textcolor[rgb]{0.10,0.09,0.49}{##1}}}
\@namedef{PY@tok@vg}{\def\PY@tc##1{\textcolor[rgb]{0.10,0.09,0.49}{##1}}}
\@namedef{PY@tok@vi}{\def\PY@tc##1{\textcolor[rgb]{0.10,0.09,0.49}{##1}}}
\@namedef{PY@tok@vm}{\def\PY@tc##1{\textcolor[rgb]{0.10,0.09,0.49}{##1}}}
\@namedef{PY@tok@sa}{\def\PY@tc##1{\textcolor[rgb]{0.73,0.13,0.13}{##1}}}
\@namedef{PY@tok@sb}{\def\PY@tc##1{\textcolor[rgb]{0.73,0.13,0.13}{##1}}}
\@namedef{PY@tok@sc}{\def\PY@tc##1{\textcolor[rgb]{0.73,0.13,0.13}{##1}}}
\@namedef{PY@tok@dl}{\def\PY@tc##1{\textcolor[rgb]{0.73,0.13,0.13}{##1}}}
\@namedef{PY@tok@s2}{\def\PY@tc##1{\textcolor[rgb]{0.73,0.13,0.13}{##1}}}
\@namedef{PY@tok@sh}{\def\PY@tc##1{\textcolor[rgb]{0.73,0.13,0.13}{##1}}}
\@namedef{PY@tok@s1}{\def\PY@tc##1{\textcolor[rgb]{0.73,0.13,0.13}{##1}}}
\@namedef{PY@tok@mb}{\def\PY@tc##1{\textcolor[rgb]{0.40,0.40,0.40}{##1}}}
\@namedef{PY@tok@mf}{\def\PY@tc##1{\textcolor[rgb]{0.40,0.40,0.40}{##1}}}
\@namedef{PY@tok@mh}{\def\PY@tc##1{\textcolor[rgb]{0.40,0.40,0.40}{##1}}}
\@namedef{PY@tok@mi}{\def\PY@tc##1{\textcolor[rgb]{0.40,0.40,0.40}{##1}}}
\@namedef{PY@tok@il}{\def\PY@tc##1{\textcolor[rgb]{0.40,0.40,0.40}{##1}}}
\@namedef{PY@tok@mo}{\def\PY@tc##1{\textcolor[rgb]{0.40,0.40,0.40}{##1}}}
\@namedef{PY@tok@ch}{\let\PY@it=\textit\def\PY@tc##1{\textcolor[rgb]{0.25,0.50,0.50}{##1}}}
\@namedef{PY@tok@cm}{\let\PY@it=\textit\def\PY@tc##1{\textcolor[rgb]{0.25,0.50,0.50}{##1}}}
\@namedef{PY@tok@cpf}{\let\PY@it=\textit\def\PY@tc##1{\textcolor[rgb]{0.25,0.50,0.50}{##1}}}
\@namedef{PY@tok@c1}{\let\PY@it=\textit\def\PY@tc##1{\textcolor[rgb]{0.25,0.50,0.50}{##1}}}
\@namedef{PY@tok@cs}{\let\PY@it=\textit\def\PY@tc##1{\textcolor[rgb]{0.25,0.50,0.50}{##1}}}

\def\PYZbs{\char`\\}
\def\PYZus{\char`\_}
\def\PYZob{\char`\{}
\def\PYZcb{\char`\}}
\def\PYZca{\char`\^}
\def\PYZam{\char`\&}
\def\PYZlt{\char`\<}
\def\PYZgt{\char`\>}
\def\PYZsh{\char`\#}
\def\PYZpc{\char`\%}
\def\PYZdl{\char`\$}
\def\PYZhy{\char`\-}
\def\PYZsq{\char`\'}
\def\PYZdq{\char`\"}
\def\PYZti{\char`\~}
% for compatibility with earlier versions
\def\PYZat{@}
\def\PYZlb{[}
\def\PYZrb{]}
\makeatother


    % For linebreaks inside Verbatim environment from package fancyvrb. 
    \makeatletter
        \newbox\Wrappedcontinuationbox 
        \newbox\Wrappedvisiblespacebox 
        \newcommand*\Wrappedvisiblespace {\textcolor{red}{\textvisiblespace}} 
        \newcommand*\Wrappedcontinuationsymbol {\textcolor{red}{\llap{\tiny$\m@th\hookrightarrow$}}} 
        \newcommand*\Wrappedcontinuationindent {3ex } 
        \newcommand*\Wrappedafterbreak {\kern\Wrappedcontinuationindent\copy\Wrappedcontinuationbox} 
        % Take advantage of the already applied Pygments mark-up to insert 
        % potential linebreaks for TeX processing. 
        %        {, <, #, %, $, ' and ": go to next line. 
        %        _, }, ^, &, >, - and ~: stay at end of broken line. 
        % Use of \textquotesingle for straight quote. 
        \newcommand*\Wrappedbreaksatspecials {% 
            \def\PYGZus{\discretionary{\char`\_}{\Wrappedafterbreak}{\char`\_}}% 
            \def\PYGZob{\discretionary{}{\Wrappedafterbreak\char`\{}{\char`\{}}% 
            \def\PYGZcb{\discretionary{\char`\}}{\Wrappedafterbreak}{\char`\}}}% 
            \def\PYGZca{\discretionary{\char`\^}{\Wrappedafterbreak}{\char`\^}}% 
            \def\PYGZam{\discretionary{\char`\&}{\Wrappedafterbreak}{\char`\&}}% 
            \def\PYGZlt{\discretionary{}{\Wrappedafterbreak\char`\<}{\char`\<}}% 
            \def\PYGZgt{\discretionary{\char`\>}{\Wrappedafterbreak}{\char`\>}}% 
            \def\PYGZsh{\discretionary{}{\Wrappedafterbreak\char`\#}{\char`\#}}% 
            \def\PYGZpc{\discretionary{}{\Wrappedafterbreak\char`\%}{\char`\%}}% 
            \def\PYGZdl{\discretionary{}{\Wrappedafterbreak\char`\$}{\char`\$}}% 
            \def\PYGZhy{\discretionary{\char`\-}{\Wrappedafterbreak}{\char`\-}}% 
            \def\PYGZsq{\discretionary{}{\Wrappedafterbreak\textquotesingle}{\textquotesingle}}% 
            \def\PYGZdq{\discretionary{}{\Wrappedafterbreak\char`\"}{\char`\"}}% 
            \def\PYGZti{\discretionary{\char`\~}{\Wrappedafterbreak}{\char`\~}}% 
        } 
        % Some characters . , ; ? ! / are not pygmentized. 
        % This macro makes them "active" and they will insert potential linebreaks 
        \newcommand*\Wrappedbreaksatpunct {% 
            \lccode`\~`\.\lowercase{\def~}{\discretionary{\hbox{\char`\.}}{\Wrappedafterbreak}{\hbox{\char`\.}}}% 
            \lccode`\~`\,\lowercase{\def~}{\discretionary{\hbox{\char`\,}}{\Wrappedafterbreak}{\hbox{\char`\,}}}% 
            \lccode`\~`\;\lowercase{\def~}{\discretionary{\hbox{\char`\;}}{\Wrappedafterbreak}{\hbox{\char`\;}}}% 
            \lccode`\~`\:\lowercase{\def~}{\discretionary{\hbox{\char`\:}}{\Wrappedafterbreak}{\hbox{\char`\:}}}% 
            \lccode`\~`\?\lowercase{\def~}{\discretionary{\hbox{\char`\?}}{\Wrappedafterbreak}{\hbox{\char`\?}}}% 
            \lccode`\~`\!\lowercase{\def~}{\discretionary{\hbox{\char`\!}}{\Wrappedafterbreak}{\hbox{\char`\!}}}% 
            \lccode`\~`\/\lowercase{\def~}{\discretionary{\hbox{\char`\/}}{\Wrappedafterbreak}{\hbox{\char`\/}}}% 
            \catcode`\.\active
            \catcode`\,\active 
            \catcode`\;\active
            \catcode`\:\active
            \catcode`\?\active
            \catcode`\!\active
            \catcode`\/\active 
            \lccode`\~`\~ 	
        }
    \makeatother

    \let\OriginalVerbatim=\Verbatim
    \makeatletter
    \renewcommand{\Verbatim}[1][1]{%
        %\parskip\z@skip
        \sbox\Wrappedcontinuationbox {\Wrappedcontinuationsymbol}%
        \sbox\Wrappedvisiblespacebox {\FV@SetupFont\Wrappedvisiblespace}%
        \def\FancyVerbFormatLine ##1{\hsize\linewidth
            \vtop{\raggedright\hyphenpenalty\z@\exhyphenpenalty\z@
                \doublehyphendemerits\z@\finalhyphendemerits\z@
                \strut ##1\strut}%
        }%
        % If the linebreak is at a space, the latter will be displayed as visible
        % space at end of first line, and a continuation symbol starts next line.
        % Stretch/shrink are however usually zero for typewriter font.
        \def\FV@Space {%
            \nobreak\hskip\z@ plus\fontdimen3\font minus\fontdimen4\font
            \discretionary{\copy\Wrappedvisiblespacebox}{\Wrappedafterbreak}
            {\kern\fontdimen2\font}%
        }%
        
        % Allow breaks at special characters using \PYG... macros.
        \Wrappedbreaksatspecials
        % Breaks at punctuation characters . , ; ? ! and / need catcode=\active 	
        \OriginalVerbatim[#1,codes*=\Wrappedbreaksatpunct]%
    }
    \makeatother

    % Exact colors from NB
    \definecolor{incolor}{HTML}{303F9F}
    \definecolor{outcolor}{HTML}{D84315}
    \definecolor{cellborder}{HTML}{CFCFCF}
    \definecolor{cellbackground}{HTML}{F7F7F7}
    
    % prompt
    \makeatletter
    \newcommand{\boxspacing}{\kern\kvtcb@left@rule\kern\kvtcb@boxsep}
    \makeatother
    \newcommand{\prompt}[4]{
        {\ttfamily\llap{{\color{#2}[#3]:\hspace{3pt}#4}}\vspace{-\baselineskip}}
    }
    

    
    % Prevent overflowing lines due to hard-to-break entities
    \sloppy 
    % Setup hyperref package
    \hypersetup{
      breaklinks=true,  % so long urls are correctly broken across lines
      colorlinks=true,
      urlcolor=urlcolor,
      linkcolor=linkcolor,
      citecolor=citecolor,
      }
    % Slightly bigger margins than the latex defaults
    
    \geometry{verbose,tmargin=1in,bmargin=1in,lmargin=1in,rmargin=1in}
    
    

\begin{document}
    
    \maketitle
\author{Thomas De Dobbeleer - Klaas Eelen - Witse Meeussen}

\section{DataScraper}

    \begin{tcolorbox}[breakable, size=fbox, boxrule=1pt, pad at break*=1mm,colback=cellbackground, colframe=cellborder]
\prompt{In}{incolor}{ }{\boxspacing}
\begin{Verbatim}[commandchars=\\\{\}]
\PY{o}{!}pip install selenium
\PY{o}{!}pip install webdriver\PYZus{}manager
\PY{o}{!}pip install BeautifulSoup4
\end{Verbatim}
\end{tcolorbox}

    \begin{tcolorbox}[breakable, size=fbox, boxrule=1pt, pad at break*=1mm,colback=cellbackground, colframe=cellborder]
\prompt{In}{incolor}{ }{\boxspacing}
\begin{Verbatim}[commandchars=\\\{\}]
\PY{k+kn}{import} \PY{n+nn}{os}
\PY{k+kn}{import} \PY{n+nn}{re}
\PY{k+kn}{from} \PY{n+nn}{selenium} \PY{k+kn}{import} \PY{n}{webdriver}
\PY{k+kn}{import} \PY{n+nn}{time}
\PY{k+kn}{from} \PY{n+nn}{webdriver\PYZus{}manager}\PY{n+nn}{.}\PY{n+nn}{chrome} \PY{k+kn}{import} \PY{n}{ChromeDriverManager}
\PY{k+kn}{from} \PY{n+nn}{bs4} \PY{k+kn}{import} \PY{n}{BeautifulSoup}
\PY{k+kn}{import} \PY{n+nn}{urllib}\PY{n+nn}{.}\PY{n+nn}{request}
\end{Verbatim}
\end{tcolorbox}

    Chrome Driver

The raw html data of a google search has a very limited amount of
images.

That is why we use a chrome driver (from selenium library) to open a
chrome browser tab and make a search on google images. Then we tell the
driver to scroll down a few times and lastly it gives us the raw html
data that is loaded.

We have tried programming around the limit that google images sets for
scrapers, but we weren't succesfull and it made it a lot slower.

Also we run this driver on a larger screen so more images would be
loaded per scroll.

    \begin{tcolorbox}[breakable, size=fbox, boxrule=1pt, pad at break*=1mm,colback=cellbackground, colframe=cellborder]
\prompt{In}{incolor}{ }{\boxspacing}
\begin{Verbatim}[commandchars=\\\{\}]
\PY{k}{def} \PY{n+nf}{getHtml}\PY{p}{(}\PY{n}{query}\PY{p}{,}\PY{n}{scrolls} \PY{o}{=} \PY{l+m+mi}{10}\PY{p}{)}\PY{p}{:}
    \PY{n}{driver} \PY{o}{=} \PY{n}{webdriver}\PY{o}{.}\PY{n}{Chrome}\PY{p}{(}\PY{n}{ChromeDriverManager}\PY{p}{(}\PY{p}{)}\PY{o}{.}\PY{n}{install}\PY{p}{(}\PY{p}{)}\PY{p}{)}

    \PY{n}{search\PYZus{}url}\PY{o}{=}\PY{l+s+s2}{\PYZdq{}}\PY{l+s+s2}{https://www.google.com/search?q=}\PY{l+s+si}{\PYZob{}q\PYZcb{}}\PY{l+s+s2}{\PYZam{}tbm=isch\PYZam{}tbs=sur}\PY{l+s+s2}{\PYZpc{}}\PY{l+s+s2}{3Afc\PYZam{}hl=en\PYZam{}ved=0CAIQpwVqFwoTCKCa1c6s4\PYZhy{}oCFQAAAAAdAAAAABAC\PYZam{}biw=1251\PYZam{}bih=568}\PY{l+s+s2}{\PYZdq{}}

    \PY{n}{driver}\PY{o}{.}\PY{n}{get}\PY{p}{(}\PY{n}{search\PYZus{}url}\PY{o}{.}\PY{n}{format}\PY{p}{(}\PY{n}{q}\PY{o}{=}\PY{n}{query}\PY{p}{)}\PY{p}{)}

    \PY{c+c1}{\PYZsh{}Scroll to the end of the page}
    \PY{n}{scrollsDone} \PY{o}{=} \PY{l+m+mi}{0}
    \PY{n}{clicked} \PY{o}{=} \PY{k+kc}{False}
    \PY{k}{while} \PY{n}{scrollsDone}\PY{o}{\PYZlt{}}\PY{n}{scrolls}\PY{p}{:}
        \PY{n+nb}{print}\PY{p}{(}\PY{n}{scrolls}\PY{p}{)}
        \PY{k}{try}\PY{p}{:}
            \PY{n}{driver}\PY{o}{.}\PY{n}{find\PYZus{}element\PYZus{}by\PYZus{}xpath}\PY{p}{(}\PY{l+s+s2}{\PYZdq{}}\PY{l+s+s2}{//span[@jsaction= }\PY{l+s+s2}{\PYZsq{}}\PY{l+s+s2}{h5M12e}\PY{l+s+s2}{\PYZsq{}}\PY{l+s+s2}{]}\PY{l+s+s2}{\PYZdq{}}\PY{p}{)}\PY{o}{.}\PY{n}{click}\PY{p}{(}\PY{p}{)}
            \PY{n}{clicked}\PY{o}{=}\PY{k+kc}{True}
            \PY{n}{time}\PY{o}{.}\PY{n}{sleep}\PY{p}{(}\PY{l+m+mi}{1}\PY{p}{)}
        \PY{k}{except}\PY{p}{:}
            \PY{n+nb}{print}\PY{p}{(}\PY{l+s+s1}{\PYZsq{}}\PY{l+s+s1}{click failed}\PY{l+s+s1}{\PYZsq{}}\PY{p}{)}
        \PY{k}{try}\PY{p}{:}
            \PY{n}{driver}\PY{o}{.}\PY{n}{find\PYZus{}element\PYZus{}by\PYZus{}xpath}\PY{p}{(}\PY{l+s+s2}{\PYZdq{}}\PY{l+s+s2}{//input[@jsaction= }\PY{l+s+s2}{\PYZsq{}}\PY{l+s+s2}{Pmjnye}\PY{l+s+s2}{\PYZsq{}}\PY{l+s+s2}{]}\PY{l+s+s2}{\PYZdq{}}\PY{p}{)}\PY{o}{.}\PY{n}{click}\PY{p}{(}\PY{p}{)}
            \PY{n}{clicked}\PY{o}{=}\PY{k+kc}{True}
            \PY{n}{time}\PY{o}{.}\PY{n}{sleep}\PY{p}{(}\PY{l+m+mi}{1}\PY{p}{)}
        \PY{k}{except}\PY{p}{:}
            \PY{n+nb}{print}\PY{p}{(}\PY{l+s+s1}{\PYZsq{}}\PY{l+s+s1}{click failed}\PY{l+s+s1}{\PYZsq{}}\PY{p}{)}
        \PY{n}{driver}\PY{o}{.}\PY{n}{execute\PYZus{}script}\PY{p}{(}\PY{l+s+s2}{\PYZdq{}}\PY{l+s+s2}{window.scrollTo(0, document.body.scrollHeight);}\PY{l+s+s2}{\PYZdq{}}\PY{p}{)}
        \PY{n}{time}\PY{o}{.}\PY{n}{sleep}\PY{p}{(}\PY{l+m+mi}{2}\PY{p}{)}\PY{c+c1}{\PYZsh{}sleep\PYZus{}between\PYZus{}interactions}
        \PY{n}{scrollsDone} \PY{o}{+}\PY{o}{=}\PY{l+m+mi}{1}
    

    \PY{c+c1}{\PYZsh{}Locate the images to be scraped from the current page }
    \PY{n}{html} \PY{o}{=} \PY{n}{driver}\PY{o}{.}\PY{n}{page\PYZus{}source}
    \PY{n}{driver}\PY{o}{.}\PY{n}{quit}
    \PY{k}{return} \PY{n}{html}
\end{Verbatim}
\end{tcolorbox}

    BeautifulSoup

We use BeautifulSoup to read and search the raw html for the right
images.

    \begin{tcolorbox}[breakable, size=fbox, boxrule=1pt, pad at break*=1mm,colback=cellbackground, colframe=cellborder]
\prompt{In}{incolor}{ }{\boxspacing}
\begin{Verbatim}[commandchars=\\\{\}]
\PY{k}{def} \PY{n+nf}{getImages}\PY{p}{(}\PY{n}{html}\PY{p}{)}\PY{p}{:}
    \PY{n}{soup} \PY{o}{=} \PY{n}{BeautifulSoup}\PY{p}{(}\PY{n}{html}\PY{p}{,}\PY{l+s+s2}{\PYZdq{}}\PY{l+s+s2}{html.parser}\PY{l+s+s2}{\PYZdq{}}\PY{p}{)}
    \PY{k}{return} \PY{n}{soup}\PY{o}{.}\PY{n}{find\PYZus{}all}\PY{p}{(}\PY{l+s+s2}{\PYZdq{}}\PY{l+s+s2}{img}\PY{l+s+s2}{\PYZdq{}} \PY{p}{,}\PY{n}{attrs}\PY{o}{=}\PY{p}{\PYZob{}}\PY{l+s+s2}{\PYZdq{}}\PY{l+s+s2}{src}\PY{l+s+s2}{\PYZdq{}}\PY{p}{:}\PY{k+kc}{True}\PY{p}{\PYZcb{}}\PY{p}{)}
\end{Verbatim}
\end{tcolorbox}

    Saving the images

This is a loop that loops through the images data and downloads the
image via the url in the src atribute.

    \begin{tcolorbox}[breakable, size=fbox, boxrule=1pt, pad at break*=1mm,colback=cellbackground, colframe=cellborder]
\prompt{In}{incolor}{ }{\boxspacing}
\begin{Verbatim}[commandchars=\\\{\}]
\PY{k}{def} \PY{n+nf}{saveImages}\PY{p}{(}\PY{n}{images}\PY{p}{,}\PY{n}{location}\PY{p}{,}\PY{n}{name} \PY{o}{=} \PY{l+s+s2}{\PYZdq{}}\PY{l+s+s2}{\PYZdq{}}\PY{p}{,}\PY{n}{count}\PY{o}{=}\PY{o}{\PYZhy{}}\PY{l+m+mi}{1}\PY{p}{)}\PY{p}{:}
    \PY{n}{number} \PY{o}{=} \PY{l+m+mi}{0}
    \PY{c+c1}{\PYZsh{}first image is the google icon}
    \PY{k}{for} \PY{n}{image} \PY{o+ow}{in} \PY{n}{images}\PY{p}{[}\PY{l+m+mi}{1}\PY{p}{:}\PY{p}{]}\PY{p}{:}
        \PY{n}{image\PYZus{}src}\PY{o}{=}\PY{n}{image}\PY{p}{[}\PY{l+s+s2}{\PYZdq{}}\PY{l+s+s2}{src}\PY{l+s+s2}{\PYZdq{}}\PY{p}{]}
        
        \PY{n}{urllib}\PY{o}{.}\PY{n}{request}\PY{o}{.}\PY{n}{urlretrieve}\PY{p}{(}\PY{n}{image\PYZus{}src}\PY{p}{,} \PY{n}{location}\PY{o}{+} \PY{n}{name} \PY{o}{+} \PY{n+nb}{str}\PY{p}{(}\PY{n}{number}\PY{p}{)}\PY{o}{+} \PY{l+s+s2}{\PYZdq{}}\PY{l+s+s2}{.png}\PY{l+s+s2}{\PYZdq{}}\PY{p}{)}
        \PY{n}{number} \PY{o}{+}\PY{o}{=} \PY{l+m+mi}{1}
        \PY{n}{count} \PY{o}{\PYZhy{}}\PY{o}{=}\PY{l+m+mi}{1}
        \PY{k}{if} \PY{n}{count} \PY{o}{==} \PY{l+m+mi}{0}\PY{p}{:}
            \PY{k}{return}
\end{Verbatim}
\end{tcolorbox}

    getData is used to to save n images for 1 query in datafolder with name
as folder and name for every image.

    \begin{tcolorbox}[breakable, size=fbox, boxrule=1pt, pad at break*=1mm,colback=cellbackground, colframe=cellborder]
\prompt{In}{incolor}{ }{\boxspacing}
\begin{Verbatim}[commandchars=\\\{\}]
\PY{k}{def} \PY{n+nf}{getData}\PY{p}{(}\PY{n}{query}\PY{p}{,}\PY{n}{n}\PY{p}{,}\PY{n}{name}\PY{p}{,}\PY{n}{datafolder}\PY{o}{=}\PY{l+s+s2}{\PYZdq{}}\PY{l+s+s2}{./Data}\PY{l+s+s2}{\PYZdq{}}\PY{p}{)}\PY{p}{:}
    \PY{n}{images} \PY{o}{=} \PY{n}{getImages}\PY{p}{(}\PY{n}{getHtml}\PY{p}{(}\PY{n}{query}\PY{p}{,}\PY{n}{scrolls}\PY{o}{=}\PY{n+nb}{round}\PY{p}{(}\PY{n}{n}\PY{o}{/}\PY{l+m+mi}{80}\PY{p}{)}\PY{o}{+}\PY{l+m+mi}{1}\PY{p}{)}\PY{p}{)}
    \PY{k}{if} \PY{o+ow}{not} \PY{n}{os}\PY{o}{.}\PY{n}{path}\PY{o}{.}\PY{n}{isdir}\PY{p}{(}\PY{n}{datafolder}\PY{o}{+}\PY{l+s+s2}{\PYZdq{}}\PY{l+s+s2}{/}\PY{l+s+s2}{\PYZdq{}}\PY{o}{+}\PY{n}{name}\PY{p}{)}\PY{p}{:} \PY{n}{os}\PY{o}{.}\PY{n}{mkdir}\PY{p}{(}\PY{n}{datafolder}\PY{o}{+}\PY{l+s+s2}{\PYZdq{}}\PY{l+s+s2}{/}\PY{l+s+s2}{\PYZdq{}}\PY{o}{+}\PY{n}{name}\PY{p}{)}
    \PY{n}{saveImages}\PY{p}{(}\PY{n}{images}\PY{p}{,}\PY{n}{location}\PY{o}{=}\PY{n}{datafolder}\PY{o}{+}\PY{l+s+s2}{\PYZdq{}}\PY{l+s+s2}{/}\PY{l+s+s2}{\PYZdq{}}\PY{o}{+}\PY{n}{name}\PY{o}{+} \PY{l+s+s2}{\PYZdq{}}\PY{l+s+s2}{/}\PY{l+s+s2}{\PYZdq{}}\PY{p}{,} \PY{n}{name}\PY{o}{=}\PY{n}{name}\PY{p}{,}\PY{n}{count}\PY{o}{=}\PY{n}{n}\PY{p}{)}
\end{Verbatim}
\end{tcolorbox}

    This function n amount of images for every query. 1000 just means max,
google images controlles the amount of images available.

    \begin{tcolorbox}[breakable, size=fbox, boxrule=1pt, pad at break*=1mm,colback=cellbackground, colframe=cellborder]
\prompt{In}{incolor}{ }{\boxspacing}
\begin{Verbatim}[commandchars=\\\{\}]
\PY{k}{def} \PY{n+nf}{getGoogleImages}\PY{p}{(}\PY{n}{querys}\PY{p}{,}\PY{n}{n}\PY{o}{=}\PY{l+m+mi}{1000}\PY{p}{,}\PY{n}{datafolder}\PY{o}{=}\PY{l+s+s2}{\PYZdq{}}\PY{l+s+s2}{./Data}\PY{l+s+s2}{\PYZdq{}}\PY{p}{)}\PY{p}{:}
    \PY{k}{if} \PY{o+ow}{not} \PY{n}{os}\PY{o}{.}\PY{n}{path}\PY{o}{.}\PY{n}{isdir}\PY{p}{(}\PY{n}{datafolder}\PY{p}{)}\PY{p}{:} \PY{n}{os}\PY{o}{.}\PY{n}{mkdir}\PY{p}{(}\PY{n}{datafolder}\PY{p}{)}
    \PY{k}{for} \PY{n}{q} \PY{o+ow}{in} \PY{n}{querys}\PY{p}{:}
        \PY{n}{getData}\PY{p}{(}\PY{n}{query}\PY{o}{=}\PY{n}{q}\PY{p}{,}\PY{n}{n}\PY{o}{=}\PY{n}{n}\PY{p}{,}\PY{n}{name}\PY{o}{=}\PY{n}{re}\PY{o}{.}\PY{n}{sub}\PY{p}{(}\PY{l+s+s2}{\PYZdq{}}\PY{l+s+s2}{ architecture}\PY{l+s+s2}{\PYZdq{}}\PY{p}{,}\PY{l+s+s2}{\PYZdq{}}\PY{l+s+s2}{\PYZdq{}}\PY{p}{,}\PY{n}{q}\PY{p}{)}\PY{p}{,}\PY{n}{datafolder}\PY{o}{=}\PY{n}{datafolder}\PY{p}{)}
\end{Verbatim}
\end{tcolorbox}

    The images saved this way are small (100x100 px). This could be an issue
if very small details are needed for categorization. But the images are
detailed enough to categorize and most categorization models use images
from 128x128, not so far of our images. Also this makes scraping the
data fast for the amount of pictures, getting higher resolution would
require a much longer downloading time. We agreed this was a fair
trade-off.

    \begin{tcolorbox}[breakable, size=fbox, boxrule=1pt, pad at break*=1mm,colback=cellbackground, colframe=cellborder]
\prompt{In}{incolor}{ }{\boxspacing}
\begin{Verbatim}[commandchars=\\\{\}]
\PY{c+c1}{\PYZsh{}\PYZdq{}art deco\PYZdq{},\PYZdq{}baroque\PYZdq{},\PYZdq{}gothic\PYZdq{},\PYZdq{}roman\PYZdq{},\PYZdq{}tudor\PYZdq{},\PYZdq{}ancient egyptian\PYZdq{} , \PYZdq{}moorisch\PYZdq{},\PYZdq{}rococo\PYZdq{},\PYZdq{}indoislamic\PYZdq{},\PYZdq{}federal\PYZdq{},\PYZdq{}expressionist\PYZdq{},\PYZdq{}modern\PYZdq{},\PYZdq{}cunstructivist\PYZdq{},}
\PY{n}{querys} \PY{o}{=} \PY{p}{[}\PY{l+s+s2}{\PYZdq{}}\PY{l+s+s2}{art deco}\PY{l+s+s2}{\PYZdq{}}\PY{p}{,}\PY{l+s+s2}{\PYZdq{}}\PY{l+s+s2}{baroque}\PY{l+s+s2}{\PYZdq{}}\PY{p}{,}\PY{l+s+s2}{\PYZdq{}}\PY{l+s+s2}{gothic}\PY{l+s+s2}{\PYZdq{}}\PY{p}{,}\PY{l+s+s2}{\PYZdq{}}\PY{l+s+s2}{roman}\PY{l+s+s2}{\PYZdq{}}\PY{p}{,}\PY{l+s+s2}{\PYZdq{}}\PY{l+s+s2}{tudor}\PY{l+s+s2}{\PYZdq{}}\PY{p}{,}\PY{l+s+s2}{\PYZdq{}}\PY{l+s+s2}{ancient egyptian}\PY{l+s+s2}{\PYZdq{}} \PY{p}{,} \PY{l+s+s2}{\PYZdq{}}\PY{l+s+s2}{moorisch}\PY{l+s+s2}{\PYZdq{}}\PY{p}{,}\PY{l+s+s2}{\PYZdq{}}\PY{l+s+s2}{rococo}\PY{l+s+s2}{\PYZdq{}}\PY{p}{,}\PY{l+s+s2}{\PYZdq{}}\PY{l+s+s2}{indoislamic}\PY{l+s+s2}{\PYZdq{}}\PY{p}{,}\PY{l+s+s2}{\PYZdq{}}\PY{l+s+s2}{federal}\PY{l+s+s2}{\PYZdq{}}\PY{p}{,}\PY{l+s+s2}{\PYZdq{}}\PY{l+s+s2}{expressionist}\PY{l+s+s2}{\PYZdq{}}\PY{p}{,}\PY{l+s+s2}{\PYZdq{}}\PY{l+s+s2}{modern}\PY{l+s+s2}{\PYZdq{}}\PY{p}{,}\PY{l+s+s2}{\PYZdq{}}\PY{l+s+s2}{cunstructivist}\PY{l+s+s2}{\PYZdq{}}\PY{p}{,}\PY{l+s+s2}{\PYZdq{}}\PY{l+s+s2}{brutalism}\PY{l+s+s2}{\PYZdq{}}\PY{p}{]}
\PY{n}{addArchitecture} \PY{o}{=} \PY{k}{lambda} \PY{n}{q}\PY{p}{:} \PY{n}{q} \PY{o}{+} \PY{l+s+s2}{\PYZdq{}}\PY{l+s+s2}{ architecture}\PY{l+s+s2}{\PYZdq{}}
\PY{n}{getGoogleImages}\PY{p}{(}\PY{n+nb}{list}\PY{p}{(}\PY{n+nb}{map}\PY{p}{(}\PY{n}{addArchitecture}\PY{p}{,}\PY{n}{querys}\PY{p}{)}\PY{p}{)}\PY{p}{,}\PY{n}{n}\PY{o}{=}\PY{l+m+mi}{20}\PY{p}{,}\PY{n}{datafolder}\PY{o}{=}\PY{l+s+s2}{\PYZdq{}}\PY{l+s+s2}{./static/data}\PY{l+s+s2}{\PYZdq{}}\PY{p}{)}
\end{Verbatim}
\end{tcolorbox}



    
\newpage
    
    

    
\section{Model training}

    Mounten van de google drive, het importeren van fastai en het uitpakken
van de foto's gebeurd in de eerste stap.

    \begin{tcolorbox}[breakable, size=fbox, boxrule=1pt, pad at break*=1mm,colback=cellbackground, colframe=cellborder]
\prompt{In}{incolor}{4}{\boxspacing}
\begin{Verbatim}[commandchars=\\\{\}]
\PY{o}{!}pip install \PYZhy{}Uqq fastbook
\PY{k+kn}{import} \PY{n+nn}{fastbook}
\PY{n}{fastbook}\PY{o}{.}\PY{n}{setup\PYZus{}book}\PY{p}{(}\PY{p}{)}
\end{Verbatim}
\end{tcolorbox}

    \begin{Verbatim}[commandchars=\\\{\}]
     || 720 kB 12.8 MB/s
     || 46 kB 3.6 MB/s
     || 189 kB 45.0 MB/s
     || 1.2 MB 49.2 MB/s
     || 56 kB 4.5 MB/s
     || 51 kB 314 kB/s
Mounted at /content/gdrive
    \end{Verbatim}

    \begin{tcolorbox}[breakable, size=fbox, boxrule=1pt, pad at break*=1mm,colback=cellbackground, colframe=cellborder]
\prompt{In}{incolor}{5}{\boxspacing}
\begin{Verbatim}[commandchars=\\\{\}]
\PY{k+kn}{from} \PY{n+nn}{fastbook} \PY{k+kn}{import} \PY{o}{*}
\PY{k+kn}{from} \PY{n+nn}{fastai}\PY{n+nn}{.}\PY{n+nn}{vision}\PY{n+nn}{.}\PY{n+nn}{widgets} \PY{k+kn}{import} \PY{o}{*}
\end{Verbatim}
\end{tcolorbox}

    \begin{tcolorbox}[breakable, size=fbox, boxrule=1pt, pad at break*=1mm,colback=cellbackground, colframe=cellborder]
\prompt{In}{incolor}{3}{\boxspacing}
\begin{Verbatim}[commandchars=\\\{\}]
\PY{o}{!}unzip data.zip
\end{Verbatim}
\end{tcolorbox}



    Het tonen van 1 foto om te zien of het pad goed is en de data goed
geladen is.

    \begin{tcolorbox}[breakable, size=fbox, boxrule=1pt, pad at break*=1mm,colback=cellbackground, colframe=cellborder]
\prompt{In}{incolor}{4}{\boxspacing}
\begin{Verbatim}[commandchars=\\\{\}]
\PY{k+kn}{from} \PY{n+nn}{PIL} \PY{k+kn}{import} \PY{n}{Image}
\PY{n}{path} \PY{o}{=} \PY{l+s+s1}{\PYZsq{}}\PY{l+s+s1}{/content/data/}\PY{l+s+s1}{\PYZsq{}}
\PY{n}{im} \PY{o}{=} \PY{n}{Image}\PY{o}{.}\PY{n}{open}\PY{p}{(}\PY{n}{path} \PY{o}{+} \PY{l+s+s1}{\PYZsq{}}\PY{l+s+s1}{art deco/art deco0.png}\PY{l+s+s1}{\PYZsq{}}\PY{p}{)}
\PY{n}{im}\PY{o}{.}\PY{n}{to\PYZus{}thumb}\PY{p}{(}\PY{l+m+mi}{128}\PY{p}{,}\PY{l+m+mi}{128}\PY{p}{)}
\end{Verbatim}
\end{tcolorbox}
 
            
\prompt{Out}{outcolor}{4}{}
    
    \begin{center}
    \adjustimage{max size={0.9\linewidth}{0.9\paperheight}}{Project_Big_Data_files/Project_Big_Data_6_0.png}
    \end{center}
    { \hspace*{\fill} \\}
    

    \begin{tcolorbox}[breakable, size=fbox, boxrule=1pt, pad at break*=1mm,colback=cellbackground, colframe=cellborder]
\prompt{In}{incolor}{5}{\boxspacing}
\begin{Verbatim}[commandchars=\\\{\}]
\PY{k+kn}{from} \PY{n+nn}{fastai}\PY{n+nn}{.}\PY{n+nn}{vision}\PY{n+nn}{.}\PY{n+nn}{all} \PY{k+kn}{import} \PY{o}{*}
\PY{n}{filenames} \PY{o}{=} \PY{n}{get\PYZus{}image\PYZus{}files}\PY{p}{(}\PY{n}{path}\PY{p}{)}
\PY{n}{filenames}
\end{Verbatim}
\end{tcolorbox}

            \begin{tcolorbox}[breakable, size=fbox, boxrule=.5pt, pad at break*=1mm, opacityfill=0]
\prompt{Out}{outcolor}{5}{\boxspacing}
\begin{Verbatim}[commandchars=\\\{\}]
(\#9053) [Path('/content/data/tudor/tudor479.png'),Path('/content/data/tudor/tudo
r653.png'),Path('/content/data/tudor/tudor114.png'),Path('/content/data/tudor/tu
dor665.png'),Path('/content/data/tudor/tudor572.png'),Path('/content/data/tudor/
tudor184.png'),Path('/content/data/tudor/tudor285.png'),Path('/content/data/tudo
r/tudor134.png'),Path('/content/data/tudor/tudor499.png'),Path('/content/data/tu
dor/tudor581.png'){\ldots}]
\end{Verbatim}
\end{tcolorbox}
        
    Nakijken of er geen corrupte files tussen staan

    \begin{tcolorbox}[breakable, size=fbox, boxrule=1pt, pad at break*=1mm,colback=cellbackground, colframe=cellborder]
\prompt{In}{incolor}{6}{\boxspacing}
\begin{Verbatim}[commandchars=\\\{\}]
\PY{n}{failed} \PY{o}{=} \PY{n}{verify\PYZus{}images}\PY{p}{(}\PY{n}{filenames}\PY{p}{)}
\PY{n}{failed}
\end{Verbatim}
\end{tcolorbox}

            \begin{tcolorbox}[breakable, size=fbox, boxrule=.5pt, pad at break*=1mm, opacityfill=0]
\prompt{Out}{outcolor}{6}{\boxspacing}
\begin{Verbatim}[commandchars=\\\{\}]
(\#0) []
\end{Verbatim}
\end{tcolorbox}
        
dataloaders


    Quick summary of the doc:

    \begin{tcolorbox}[breakable, size=fbox, boxrule=1pt, pad at break*=1mm,colback=cellbackground, colframe=cellborder]
\prompt{In}{incolor}{6}{\boxspacing}
\begin{Verbatim}[commandchars=\\\{\}]
\PY{o}{??}DataLoaders
\end{Verbatim}
\end{tcolorbox}

fastai-data-block-api


    \begin{tcolorbox}[breakable, size=fbox, boxrule=1pt, pad at break*=1mm,colback=cellbackground, colframe=cellborder]
\prompt{In}{incolor}{8}{\boxspacing}
\begin{Verbatim}[commandchars=\\\{\}]
\PY{n}{architecture} \PY{o}{=} \PY{n}{DataBlock}\PY{p}{(}
    \PY{n}{blocks}\PY{o}{=}\PY{p}{(}\PY{n}{ImageBlock}\PY{p}{,} \PY{n}{CategoryBlock}\PY{p}{)}\PY{p}{,} 
    \PY{n}{get\PYZus{}items}\PY{o}{=}\PY{n}{get\PYZus{}image\PYZus{}files}\PY{p}{,} 
    \PY{n}{splitter}\PY{o}{=}\PY{n}{RandomSplitter}\PY{p}{(}\PY{n}{valid\PYZus{}pct}\PY{o}{=}\PY{l+m+mf}{0.2}\PY{p}{,} \PY{n}{seed}\PY{o}{=}\PY{l+m+mi}{42}\PY{p}{)}\PY{p}{,}
    \PY{n}{get\PYZus{}y}\PY{o}{=}\PY{n}{parent\PYZus{}label}\PY{p}{,}
    \PY{n}{item\PYZus{}tfms}\PY{o}{=}\PY{n}{Resize}\PY{p}{(}\PY{l+m+mi}{128}\PY{p}{)}\PY{p}{)}
\end{Verbatim}
\end{tcolorbox}

    \begin{tcolorbox}[breakable, size=fbox, boxrule=1pt, pad at break*=1mm,colback=cellbackground, colframe=cellborder]
\prompt{In}{incolor}{9}{\boxspacing}
\begin{Verbatim}[commandchars=\\\{\}]
\PY{n}{doc}\PY{p}{(}\PY{n}{DataBlock}\PY{p}{)}
\end{Verbatim}
\end{tcolorbox}

    
    \begin{Verbatim}[commandchars=\\\{\}]
<IPython.core.display.HTML object>
    \end{Verbatim}

    
    \begin{tcolorbox}[breakable, size=fbox, boxrule=1pt, pad at break*=1mm,colback=cellbackground, colframe=cellborder]
\prompt{In}{incolor}{11}{\boxspacing}
\begin{Verbatim}[commandchars=\\\{\}]
\PY{n}{dls} \PY{o}{=} \PY{n}{architecture}\PY{o}{.}\PY{n}{dataloaders}\PY{p}{(}\PY{n}{path}\PY{p}{)}
\end{Verbatim}
\end{tcolorbox}

    \begin{tcolorbox}[breakable, size=fbox, boxrule=1pt, pad at break*=1mm,colback=cellbackground, colframe=cellborder]
\prompt{In}{incolor}{12}{\boxspacing}
\begin{Verbatim}[commandchars=\\\{\}]
\PY{n}{dls}\PY{o}{.}\PY{n}{train}\PY{o}{.}\PY{n}{show\PYZus{}batch}\PY{p}{(}\PY{n}{max\PYZus{}n}\PY{o}{=}\PY{l+m+mi}{4}\PY{p}{,} \PY{n}{nrows}\PY{o}{=}\PY{l+m+mi}{1}\PY{p}{)}
\PY{n}{dls}\PY{o}{.}\PY{n}{valid}\PY{o}{.}\PY{n}{show\PYZus{}batch}\PY{p}{(}\PY{n}{max\PYZus{}n}\PY{o}{=}\PY{l+m+mi}{4}\PY{p}{,} \PY{n}{nrows}\PY{o}{=}\PY{l+m+mi}{1}\PY{p}{)}
\end{Verbatim}
\end{tcolorbox}

    \begin{center}
    \adjustimage{max size={0.9\linewidth}{0.9\paperheight}}{Project_Big_Data_files/Project_Big_Data_17_0.png}
    \end{center}
    { \hspace*{\fill} \\}
    
    \begin{center}
    \adjustimage{max size={0.9\linewidth}{0.9\paperheight}}{Project_Big_Data_files/Project_Big_Data_17_1.png}
    \end{center}
    { \hspace*{\fill} \\}
    
    Tonen van 8 foto's 4 per row die ge resized zijn

    \begin{tcolorbox}[breakable, size=fbox, boxrule=1pt, pad at break*=1mm,colback=cellbackground, colframe=cellborder]
\prompt{In}{incolor}{13}{\boxspacing}
\begin{Verbatim}[commandchars=\\\{\}]
\PY{n}{architecture} \PY{o}{=} \PY{n}{architecture}\PY{o}{.}\PY{n}{new}\PY{p}{(}\PY{n}{item\PYZus{}tfms}\PY{o}{=}\PY{n}{Resize}\PY{p}{(}\PY{l+m+mi}{128}\PY{p}{,} \PY{n}{ResizeMethod}\PY{o}{.}\PY{n}{Squish}\PY{p}{)}\PY{p}{)}
\PY{n}{dls} \PY{o}{=} \PY{n}{architecture}\PY{o}{.}\PY{n}{dataloaders}\PY{p}{(}\PY{n}{path}\PY{p}{)}
\PY{n}{dls}\PY{o}{.}\PY{n}{train}\PY{o}{.}\PY{n}{show\PYZus{}batch}\PY{p}{(}\PY{n}{max\PYZus{}n}\PY{o}{=}\PY{l+m+mi}{4}\PY{p}{,} \PY{n}{nrows}\PY{o}{=}\PY{l+m+mi}{1}\PY{p}{)}
\PY{n}{dls}\PY{o}{.}\PY{n}{valid}\PY{o}{.}\PY{n}{show\PYZus{}batch}\PY{p}{(}\PY{n}{max\PYZus{}n}\PY{o}{=}\PY{l+m+mi}{4}\PY{p}{,} \PY{n}{nrows}\PY{o}{=}\PY{l+m+mi}{1}\PY{p}{)}
\end{Verbatim}
\end{tcolorbox}

    \begin{center}
    \adjustimage{max size={0.9\linewidth}{0.9\paperheight}}{Project_Big_Data_files/Project_Big_Data_19_0.png}
    \end{center}
    { \hspace*{\fill} \\}
    
    \begin{center}
    \adjustimage{max size={0.9\linewidth}{0.9\paperheight}}{Project_Big_Data_files/Project_Big_Data_19_1.png}
    \end{center}
    { \hspace*{\fill} \\}
    
    \begin{tcolorbox}[breakable, size=fbox, boxrule=1pt, pad at break*=1mm,colback=cellbackground, colframe=cellborder]
\prompt{In}{incolor}{14}{\boxspacing}
\begin{Verbatim}[commandchars=\\\{\}]
\PY{n}{architecture} \PY{o}{=} \PY{n}{architecture}\PY{o}{.}\PY{n}{new}\PY{p}{(}\PY{n}{item\PYZus{}tfms}\PY{o}{=}\PY{n}{Resize}\PY{p}{(}\PY{l+m+mi}{128}\PY{p}{,} \PY{n}{ResizeMethod}\PY{o}{.}\PY{n}{Pad}\PY{p}{,} \PY{n}{pad\PYZus{}mode}\PY{o}{=}\PY{l+s+s1}{\PYZsq{}}\PY{l+s+s1}{zeros}\PY{l+s+s1}{\PYZsq{}}\PY{p}{)}\PY{p}{)}
\PY{n}{dls} \PY{o}{=} \PY{n}{architecture}\PY{o}{.}\PY{n}{dataloaders}\PY{p}{(}\PY{n}{path}\PY{p}{)}
\PY{n}{dls}\PY{o}{.}\PY{n}{train}\PY{o}{.}\PY{n}{show\PYZus{}batch}\PY{p}{(}\PY{n}{max\PYZus{}n}\PY{o}{=}\PY{l+m+mi}{4}\PY{p}{,} \PY{n}{nrows}\PY{o}{=}\PY{l+m+mi}{1}\PY{p}{)}
\PY{n}{dls}\PY{o}{.}\PY{n}{valid}\PY{o}{.}\PY{n}{show\PYZus{}batch}\PY{p}{(}\PY{n}{max\PYZus{}n}\PY{o}{=}\PY{l+m+mi}{4}\PY{p}{,} \PY{n}{nrows}\PY{o}{=}\PY{l+m+mi}{1}\PY{p}{)}
\end{Verbatim}
\end{tcolorbox}

    \begin{center}
    \adjustimage{max size={0.9\linewidth}{0.9\paperheight}}{Project_Big_Data_files/Project_Big_Data_20_0.png}
    \end{center}
    { \hspace*{\fill} \\}
    
    \begin{center}
    \adjustimage{max size={0.9\linewidth}{0.9\paperheight}}{Project_Big_Data_files/Project_Big_Data_20_1.png}
    \end{center}
    { \hspace*{\fill} \\}
    
    \begin{tcolorbox}[breakable, size=fbox, boxrule=1pt, pad at break*=1mm,colback=cellbackground, colframe=cellborder]
\prompt{In}{incolor}{15}{\boxspacing}
\begin{Verbatim}[commandchars=\\\{\}]
\PY{n}{architecture} \PY{o}{=} \PY{n}{architecture}\PY{o}{.}\PY{n}{new}\PY{p}{(}\PY{n}{item\PYZus{}tfms}\PY{o}{=}\PY{n}{RandomResizedCrop}\PY{p}{(}\PY{l+m+mi}{128}\PY{p}{,} \PY{n}{min\PYZus{}scale}\PY{o}{=}\PY{l+m+mf}{0.3}\PY{p}{)}\PY{p}{)}
\PY{n}{dls} \PY{o}{=} \PY{n}{architecture}\PY{o}{.}\PY{n}{dataloaders}\PY{p}{(}\PY{n}{path}\PY{p}{)}
\PY{n}{dls}\PY{o}{.}\PY{n}{train}\PY{o}{.}\PY{n}{show\PYZus{}batch}\PY{p}{(}\PY{n}{max\PYZus{}n}\PY{o}{=}\PY{l+m+mi}{4}\PY{p}{,} \PY{n}{nrows}\PY{o}{=}\PY{l+m+mi}{1}\PY{p}{,} \PY{n}{unique}\PY{o}{=}\PY{k+kc}{True}\PY{p}{)}
\PY{c+c1}{\PYZsh{} the unique=True piece here is just to force our sanity check to give us the same image over and over again, }
\PY{c+c1}{\PYZsh{} ofcourse each time with the random transform applied to it}
\end{Verbatim}
\end{tcolorbox}

    \begin{center}
    \adjustimage{max size={0.9\linewidth}{0.9\paperheight}}{Project_Big_Data_files/Project_Big_Data_21_0.png}
    \end{center}
    { \hspace*{\fill} \\}
    
    Het trainen van het model kan eindelijk gebeuren. We hebben gekozen voor
maar 3 epochs aangezien vanaf 4 epochs de data biased werd.

    \begin{tcolorbox}[breakable, size=fbox, boxrule=1pt, pad at break*=1mm,colback=cellbackground, colframe=cellborder]
\prompt{In}{incolor}{18}{\boxspacing}
\begin{Verbatim}[commandchars=\\\{\}]
\PY{n}{our\PYZus{}out\PYZus{}of\PYZus{}the\PYZus{}box\PYZus{}model} \PY{o}{=} \PY{n}{cnn\PYZus{}learner}\PY{p}{(}\PY{n}{dls}\PY{p}{,} \PY{n}{resnet50}\PY{p}{,} \PY{n}{metrics}\PY{o}{=}\PY{n}{error\PYZus{}rate}\PY{p}{)}
\PY{n}{our\PYZus{}out\PYZus{}of\PYZus{}the\PYZus{}box\PYZus{}model}\PY{o}{.}\PY{n}{fine\PYZus{}tune}\PY{p}{(}\PY{l+m+mi}{3}\PY{p}{)}
\end{Verbatim}
\end{tcolorbox}

    
    \begin{Verbatim}[commandchars=\\\{\}]
<IPython.core.display.HTML object>
    \end{Verbatim}

    
    
    \begin{Verbatim}[commandchars=\\\{\}]
<IPython.core.display.HTML object>
    \end{Verbatim}

    
    \begin{tcolorbox}[breakable, size=fbox, boxrule=1pt, pad at break*=1mm,colback=cellbackground, colframe=cellborder]
\prompt{In}{incolor}{19}{\boxspacing}
\begin{Verbatim}[commandchars=\\\{\}]
\PY{n}{interp} \PY{o}{=} \PY{n}{ClassificationInterpretation}\PY{o}{.}\PY{n}{from\PYZus{}learner}\PY{p}{(}\PY{n}{our\PYZus{}out\PYZus{}of\PYZus{}the\PYZus{}box\PYZus{}model}\PY{p}{)}
\PY{n}{interp}\PY{o}{.}\PY{n}{plot\PYZus{}confusion\PYZus{}matrix}\PY{p}{(}\PY{p}{)}
\end{Verbatim}
\end{tcolorbox}

    
    \begin{Verbatim}[commandchars=\\\{\}]
<IPython.core.display.HTML object>
    \end{Verbatim}

    
    \begin{center}
    \adjustimage{max size={0.9\linewidth}{0.9\paperheight}}{Project_Big_Data_files/Project_Big_Data_24_1.png}
    \end{center}
    { \hspace*{\fill} \\}
    
google-teachable-machine-verwarringsmatrix

\begin{figure}
\centering
\caption{yTdnY5mYxwwAAAAAElFTkSuQmCC.png}
\end{figure}

    \begin{tcolorbox}[breakable, size=fbox, boxrule=1pt, pad at break*=1mm,colback=cellbackground, colframe=cellborder]
\prompt{In}{incolor}{20}{\boxspacing}
\begin{Verbatim}[commandchars=\\\{\}]
\PY{c+c1}{\PYZsh{}interp.plot\PYZus{}top\PYZus{}losses(2, nrows=1)}

\PY{k}{def} \PY{n+nf}{plot\PYZus{}top\PYZus{}losses\PYZus{}fix}\PY{p}{(}\PY{n}{interp}\PY{p}{,} \PY{n}{k}\PY{p}{,} \PY{n}{largest}\PY{o}{=}\PY{k+kc}{True}\PY{p}{,} \PY{o}{*}\PY{o}{*}\PY{n}{kwargs}\PY{p}{)}\PY{p}{:}
        \PY{n}{losses}\PY{p}{,}\PY{n}{idx} \PY{o}{=} \PY{n}{interp}\PY{o}{.}\PY{n}{top\PYZus{}losses}\PY{p}{(}\PY{n}{k}\PY{p}{,} \PY{n}{largest}\PY{p}{)}
        \PY{k}{if} \PY{o+ow}{not} \PY{n+nb}{isinstance}\PY{p}{(}\PY{n}{interp}\PY{o}{.}\PY{n}{inputs}\PY{p}{,} \PY{n+nb}{tuple}\PY{p}{)}\PY{p}{:} \PY{n}{interp}\PY{o}{.}\PY{n}{inputs} \PY{o}{=} \PY{p}{(}\PY{n}{interp}\PY{o}{.}\PY{n}{inputs}\PY{p}{,}\PY{p}{)}
        \PY{k}{if} \PY{n+nb}{isinstance}\PY{p}{(}\PY{n}{interp}\PY{o}{.}\PY{n}{inputs}\PY{p}{[}\PY{l+m+mi}{0}\PY{p}{]}\PY{p}{,} \PY{n}{Tensor}\PY{p}{)}\PY{p}{:} \PY{n}{inps} \PY{o}{=} \PY{n+nb}{tuple}\PY{p}{(}\PY{n}{o}\PY{p}{[}\PY{n}{idx}\PY{p}{]} \PY{k}{for} \PY{n}{o} \PY{o+ow}{in} \PY{n}{interp}\PY{o}{.}\PY{n}{inputs}\PY{p}{)}
        \PY{k}{else}\PY{p}{:} \PY{n}{inps} \PY{o}{=} \PY{n}{interp}\PY{o}{.}\PY{n}{dl}\PY{o}{.}\PY{n}{create\PYZus{}batch}\PY{p}{(}\PY{n}{interp}\PY{o}{.}\PY{n}{dl}\PY{o}{.}\PY{n}{before\PYZus{}batch}\PY{p}{(}\PY{p}{[}\PY{n+nb}{tuple}\PY{p}{(}\PY{n}{o}\PY{p}{[}\PY{n}{i}\PY{p}{]} \PY{k}{for} \PY{n}{o} \PY{o+ow}{in} \PY{n}{interp}\PY{o}{.}\PY{n}{inputs}\PY{p}{)} \PY{k}{for} \PY{n}{i} \PY{o+ow}{in} \PY{n}{idx}\PY{p}{]}\PY{p}{)}\PY{p}{)}
        \PY{n}{b} \PY{o}{=} \PY{n}{inps} \PY{o}{+} \PY{n+nb}{tuple}\PY{p}{(}\PY{n}{o}\PY{p}{[}\PY{n}{idx}\PY{p}{]} \PY{k}{for} \PY{n}{o} \PY{o+ow}{in} \PY{p}{(}\PY{n}{interp}\PY{o}{.}\PY{n}{targs} \PY{k}{if} \PY{n}{is\PYZus{}listy}\PY{p}{(}\PY{n}{interp}\PY{o}{.}\PY{n}{targs}\PY{p}{)} \PY{k}{else} \PY{p}{(}\PY{n}{interp}\PY{o}{.}\PY{n}{targs}\PY{p}{,}\PY{p}{)}\PY{p}{)}\PY{p}{)}
        \PY{n}{x}\PY{p}{,}\PY{n}{y}\PY{p}{,}\PY{n}{its} \PY{o}{=} \PY{n}{interp}\PY{o}{.}\PY{n}{dl}\PY{o}{.}\PY{n}{\PYZus{}pre\PYZus{}show\PYZus{}batch}\PY{p}{(}\PY{n}{b}\PY{p}{,} \PY{n}{max\PYZus{}n}\PY{o}{=}\PY{n}{k}\PY{p}{)}
        \PY{n}{b\PYZus{}out} \PY{o}{=} \PY{n}{inps} \PY{o}{+} \PY{n+nb}{tuple}\PY{p}{(}\PY{n}{o}\PY{p}{[}\PY{n}{idx}\PY{p}{]} \PY{k}{for} \PY{n}{o} \PY{o+ow}{in} \PY{p}{(}\PY{n}{interp}\PY{o}{.}\PY{n}{decoded} \PY{k}{if} \PY{n}{is\PYZus{}listy}\PY{p}{(}\PY{n}{interp}\PY{o}{.}\PY{n}{decoded}\PY{p}{)} \PY{k}{else} \PY{p}{(}\PY{n}{interp}\PY{o}{.}\PY{n}{decoded}\PY{p}{,}\PY{p}{)}\PY{p}{)}\PY{p}{)}
        \PY{n}{x1}\PY{p}{,}\PY{n}{y1}\PY{p}{,}\PY{n}{outs} \PY{o}{=} \PY{n}{interp}\PY{o}{.}\PY{n}{dl}\PY{o}{.}\PY{n}{\PYZus{}pre\PYZus{}show\PYZus{}batch}\PY{p}{(}\PY{n}{b\PYZus{}out}\PY{p}{,} \PY{n}{max\PYZus{}n}\PY{o}{=}\PY{n}{k}\PY{p}{)}
        \PY{k}{if} \PY{n}{its} \PY{o+ow}{is} \PY{o+ow}{not} \PY{k+kc}{None}\PY{p}{:}
            \PY{c+c1}{\PYZsh{}plot\PYZus{}top\PYZus{}losses(x, y, its, outs.itemgot(slice(len(inps), None)), L(self.preds).itemgot(idx), losses,  **kwargs)}
            \PY{n}{plot\PYZus{}top\PYZus{}losses}\PY{p}{(}\PY{n}{x}\PY{p}{,} \PY{n}{y}\PY{p}{,} \PY{n}{its}\PY{p}{,} \PY{n}{outs}\PY{o}{.}\PY{n}{itemgot}\PY{p}{(}\PY{n+nb}{slice}\PY{p}{(}\PY{n+nb}{len}\PY{p}{(}\PY{n}{inps}\PY{p}{)}\PY{p}{,} \PY{k+kc}{None}\PY{p}{)}\PY{p}{)}\PY{p}{,} \PY{n}{interp}\PY{o}{.}\PY{n}{preds}\PY{p}{[}\PY{n}{idx}\PY{p}{]}\PY{p}{,} \PY{n}{losses}\PY{p}{,}  \PY{o}{*}\PY{o}{*}\PY{n}{kwargs}\PY{p}{)}
        \PY{c+c1}{\PYZsh{}TODO: figure out if this is needed}
        \PY{c+c1}{\PYZsh{}its None means that a batch knows how to show itself as a whole, so we pass x, x1}
        \PY{c+c1}{\PYZsh{}else: show\PYZus{}results(x, x1, its, ctxs=ctxs, max\PYZus{}n=max\PYZus{}n, **kwargs)}
\end{Verbatim}
\end{tcolorbox}

    \begin{tcolorbox}[breakable, size=fbox, boxrule=1pt, pad at break*=1mm,colback=cellbackground, colframe=cellborder]
\prompt{In}{incolor}{21}{\boxspacing}
\begin{Verbatim}[commandchars=\\\{\}]
\PY{n}{plot\PYZus{}top\PYZus{}losses\PYZus{}fix}\PY{p}{(}\PY{n}{interp}\PY{p}{,} \PY{l+m+mi}{10}\PY{p}{,} \PY{n}{nrows}\PY{o}{=}\PY{l+m+mi}{2}\PY{p}{)}
\end{Verbatim}
\end{tcolorbox}

    \begin{center}
    \adjustimage{max size={0.9\linewidth}{0.9\paperheight}}{Project_Big_Data_files/Project_Big_Data_27_0.png}
    \end{center}
    { \hspace*{\fill} \\}
    
    \begin{tcolorbox}[breakable, size=fbox, boxrule=1pt, pad at break*=1mm,colback=cellbackground, colframe=cellborder]
\prompt{In}{incolor}{22}{\boxspacing}
\begin{Verbatim}[commandchars=\\\{\}]
\PY{c+c1}{\PYZsh{} saving our model, by default in a folder called \PYZsq{}models\PYZsq{}.}
\PY{n}{our\PYZus{}out\PYZus{}of\PYZus{}the\PYZus{}box\PYZus{}model}\PY{o}{.}\PY{n}{save}\PY{p}{(}\PY{l+s+s1}{\PYZsq{}}\PY{l+s+s1}{good\PYZus{}model}\PY{l+s+s1}{\PYZsq{}}\PY{p}{)}
\PY{c+c1}{\PYZsh{}creating an serialized pickle object of our model, the export.pkl file}
\PY{n}{our\PYZus{}out\PYZus{}of\PYZus{}the\PYZus{}box\PYZus{}model}\PY{o}{.}\PY{n}{export}\PY{p}{(}\PY{p}{)}
\end{Verbatim}
\end{tcolorbox}

    \begin{tcolorbox}[breakable, size=fbox, boxrule=1pt, pad at break*=1mm,colback=cellbackground, colframe=cellborder]
\prompt{In}{incolor}{22}{\boxspacing}
\begin{Verbatim}[commandchars=\\\{\}]
\PY{n}{ls}
\end{Verbatim}
\end{tcolorbox}

    \begin{Verbatim}[commandchars=\\\{\}]
\textcolor{ansi-blue-intense}{\textbf{data}}/  data.zip  export.pkl  \textcolor{ansi-blue-intense}{\textbf{gdrive}}/
\textcolor{ansi-blue-intense}{\textbf{models}}/  \textcolor{ansi-blue-intense}{\textbf{sample\_data}}/
    \end{Verbatim}

loading-a-model-inference

    \begin{tcolorbox}[breakable, size=fbox, boxrule=1pt, pad at break*=1mm,colback=cellbackground, colframe=cellborder]
\prompt{In}{incolor}{24}{\boxspacing}
\begin{Verbatim}[commandchars=\\\{\}]
\PY{n}{our\PYZus{}out\PYZus{}of\PYZus{}the\PYZus{}box\PYZus{}model\PYZus{}inference} \PY{o}{=} \PY{n}{load\PYZus{}learner}\PY{p}{(}\PY{l+s+s1}{\PYZsq{}}\PY{l+s+s1}{export.pkl}\PY{l+s+s1}{\PYZsq{}}\PY{p}{)}
\PY{c+c1}{\PYZsh{} let\PYZsq{}s test our model on an image}
\PY{n}{our\PYZus{}out\PYZus{}of\PYZus{}the\PYZus{}box\PYZus{}model\PYZus{}inference}\PY{o}{.}\PY{n}{predict}\PY{p}{(}\PY{l+s+s1}{\PYZsq{}}\PY{l+s+s1}{/content/data/modern/modern101.png}\PY{l+s+s1}{\PYZsq{}}\PY{p}{)}
\PY{c+c1}{\PYZsh{} this will return the predicted category, the index of this predicted category, and the probabilities of each category}
\end{Verbatim}
\end{tcolorbox}

    
    \begin{Verbatim}[commandchars=\\\{\}]
<IPython.core.display.HTML object>
    \end{Verbatim}

    
            \begin{tcolorbox}[breakable, size=fbox, boxrule=.5pt, pad at break*=1mm, opacityfill=0]
\prompt{Out}{outcolor}{24}{\boxspacing}
\begin{Verbatim}[commandchars=\\\{\}]
('expressionist',
 TensorBase(5),
 TensorBase([4.7753e-03, 2.5152e-02, 5.1793e-04, 4.0738e-02, 4.2338e-03,
6.1465e-01, 4.5408e-02, 2.2867e-03, 9.1173e-03, 2.5046e-01, 6.2986e-04,
5.1209e-04, 1.3096e-03, 2.1098e-04]))
\end{Verbatim}
\end{tcolorbox}
        
    \begin{tcolorbox}[breakable, size=fbox, boxrule=1pt, pad at break*=1mm,colback=cellbackground, colframe=cellborder]
\prompt{In}{incolor}{ }{\boxspacing}
\begin{Verbatim}[commandchars=\\\{\}]
\PY{n}{our\PYZus{}out\PYZus{}of\PYZus{}the\PYZus{}box\PYZus{}model\PYZus{}inference}\PY{o}{.}\PY{n}{dls}\PY{o}{.}\PY{n}{vocab}
\end{Verbatim}
\end{tcolorbox}

    \begin{center}\rule{0.5\linewidth}{0.5pt}\end{center}

    \begin{tcolorbox}[breakable, size=fbox, boxrule=1pt, pad at break*=1mm,colback=cellbackground, colframe=cellborder]
\prompt{In}{incolor}{1}{\boxspacing}
\begin{Verbatim}[commandchars=\\\{\}]
\PY{o}{!}pip install \PYZhy{}Uqq fastbook
\PY{k+kn}{import} \PY{n+nn}{fastbook}
\PY{n}{fastbook}\PY{o}{.}\PY{n}{setup\PYZus{}book}\PY{p}{(}\PY{p}{)}
\end{Verbatim}
\end{tcolorbox}

    \begin{Verbatim}[commandchars=\\\{\}]
     || 720 kB 5.1 MB/s
     || 189 kB 45.7 MB/s
     || 46 kB 3.9 MB/s
     || 1.2 MB 36.0 MB/s
     || 56 kB 4.4 MB/s
     || 51 kB 286 kB/s
Mounted at /content/gdrive
    \end{Verbatim}

    \begin{tcolorbox}[breakable, size=fbox, boxrule=1pt, pad at break*=1mm,colback=cellbackground, colframe=cellborder]
\prompt{In}{incolor}{2}{\boxspacing}
\begin{Verbatim}[commandchars=\\\{\}]
\PY{k+kn}{from} \PY{n+nn}{fastbook} \PY{k+kn}{import} \PY{o}{*}
\end{Verbatim}
\end{tcolorbox}

    \begin{tcolorbox}[breakable, size=fbox, boxrule=1pt, pad at break*=1mm,colback=cellbackground, colframe=cellborder]
\prompt{In}{incolor}{3}{\boxspacing}
\begin{Verbatim}[commandchars=\\\{\}]
\PY{o}{!}unzip data.zip
\end{Verbatim}
\end{tcolorbox}



    \begin{tcolorbox}[breakable, size=fbox, boxrule=1pt, pad at break*=1mm,colback=cellbackground, colframe=cellborder]
\prompt{In}{incolor}{10}{\boxspacing}
\begin{Verbatim}[commandchars=\\\{\}]
\PY{n}{Gebouwen} \PY{o}{=} \PY{n}{DataBlock}\PY{p}{(}
    \PY{n}{blocks}\PY{o}{=}\PY{p}{(}\PY{n}{ImageBlock}\PY{p}{,} \PY{n}{CategoryBlock}\PY{p}{)}\PY{p}{,} 
    \PY{n}{get\PYZus{}items}\PY{o}{=}\PY{n}{get\PYZus{}image\PYZus{}files}\PY{p}{,} 
    \PY{n}{splitter}\PY{o}{=}\PY{n}{RandomSplitter}\PY{p}{(}\PY{n}{valid\PYZus{}pct}\PY{o}{=}\PY{l+m+mf}{0.2}\PY{p}{,} \PY{n}{seed}\PY{o}{=}\PY{l+m+mi}{42}\PY{p}{)}\PY{p}{,}
    \PY{c+c1}{\PYZsh{} the line of code below is just using Regular Expressions to link a file to a label}
    \PY{c+c1}{\PYZsh{} the label, the breed of the pet, is in the filename, that\PYZsq{}s why we need a re to extract it as the label }
    \PY{n}{get\PYZus{}y}\PY{o}{=}\PY{n}{parent\PYZus{}label}\PY{p}{,}
      \PY{c+c1}{\PYZsh{} now let\PYZsq{}s also add some awesome augmentation (presizing) into the mix as well}
    \PY{n}{item\PYZus{}tfms}\PY{o}{=}\PY{n}{Resize}\PY{p}{(}\PY{l+m+mi}{460}\PY{p}{)}\PY{p}{,}
    \PY{n}{batch\PYZus{}tfms}\PY{o}{=}\PY{n}{aug\PYZus{}transforms}\PY{p}{(}\PY{n}{size}\PY{o}{=}\PY{l+m+mi}{224}\PY{p}{,} \PY{n}{min\PYZus{}scale}\PY{o}{=}\PY{l+m+mf}{0.75}\PY{p}{)}\PY{p}{)}
\end{Verbatim}
\end{tcolorbox}

    \begin{tcolorbox}[breakable, size=fbox, boxrule=1pt, pad at break*=1mm,colback=cellbackground, colframe=cellborder]
\prompt{In}{incolor}{11}{\boxspacing}
\begin{Verbatim}[commandchars=\\\{\}]
\PY{n}{path} \PY{o}{=} \PY{l+s+s1}{\PYZsq{}}\PY{l+s+s1}{/content/data/}\PY{l+s+s1}{\PYZsq{}}
\PY{n}{dls} \PY{o}{=} \PY{n}{Gebouwen}\PY{o}{.}\PY{n}{dataloaders}\PY{p}{(}\PY{n}{path}\PY{p}{)}
\end{Verbatim}
\end{tcolorbox}

    \begin{Verbatim}[commandchars=\\\{\}]
/usr/local/lib/python3.7/dist-packages/torch/\_tensor.py:1051: UserWarning:
torch.solve is deprecated in favor of torch.linalg.solveand will be removed in a
future PyTorch release.
torch.linalg.solve has its arguments reversed and does not return the LU
factorization.
To get the LU factorization see torch.lu, which can be used with torch.lu\_solve
or torch.lu\_unpack.
X = torch.solve(B, A).solution
should be replaced with
X = torch.linalg.solve(A, B) (Triggered internally at
../aten/src/ATen/native/BatchLinearAlgebra.cpp:766.)
  ret = func(*args, **kwargs)
    \end{Verbatim}

    \begin{tcolorbox}[breakable, size=fbox, boxrule=1pt, pad at break*=1mm,colback=cellbackground, colframe=cellborder]
\prompt{In}{incolor}{12}{\boxspacing}
\begin{Verbatim}[commandchars=\\\{\}]
\PY{n}{dls}\PY{o}{.}\PY{n}{train}\PY{o}{.}\PY{n}{show\PYZus{}batch}\PY{p}{(}\PY{n}{max\PYZus{}n}\PY{o}{=}\PY{l+m+mi}{4}\PY{p}{,} \PY{n}{nrows}\PY{o}{=}\PY{l+m+mi}{1}\PY{p}{)}
\PY{n}{dls}\PY{o}{.}\PY{n}{valid}\PY{o}{.}\PY{n}{show\PYZus{}batch}\PY{p}{(}\PY{n}{max\PYZus{}n}\PY{o}{=}\PY{l+m+mi}{4}\PY{p}{,} \PY{n}{nrows}\PY{o}{=}\PY{l+m+mi}{1}\PY{p}{)}
\end{Verbatim}
\end{tcolorbox}

    \begin{center}
    \adjustimage{max size={0.9\linewidth}{0.9\paperheight}}{Project_Big_Data_files/Project_Big_Data_39_0.png}
    \end{center}
    { \hspace*{\fill} \\}
    
    \begin{center}
    \adjustimage{max size={0.9\linewidth}{0.9\paperheight}}{Project_Big_Data_files/Project_Big_Data_39_1.png}
    \end{center}
    { \hspace*{\fill} \\}
    
    \begin{tcolorbox}[breakable, size=fbox, boxrule=1pt, pad at break*=1mm,colback=cellbackground, colframe=cellborder]
\prompt{In}{incolor}{13}{\boxspacing}
\begin{Verbatim}[commandchars=\\\{\}]
\PY{n}{alex\PYZus{}model} \PY{o}{=} \PY{n}{cnn\PYZus{}learner}\PY{p}{(}\PY{n}{dls}\PY{p}{,} \PY{n}{alexnet}\PY{p}{,} \PY{n}{metrics}\PY{o}{=}\PY{n}{error\PYZus{}rate}\PY{p}{)}
\PY{n}{alex\PYZus{}model}\PY{o}{.}\PY{n}{fine\PYZus{}tune}\PY{p}{(}\PY{l+m+mi}{1}\PY{p}{)}
\end{Verbatim}
\end{tcolorbox}

    \begin{Verbatim}[commandchars=\\\{\}]
Downloading: "https://download.pytorch.org/models/alexnet-owt-7be5be79.pth" to
/root/.cache/torch/hub/checkpoints/alexnet-owt-7be5be79.pth
    \end{Verbatim}

    
    \begin{Verbatim}[commandchars=\\\{\}]
  0\%|          | 0.00/233M [00:00<?, ?B/s]
    \end{Verbatim}

    
    
    \begin{Verbatim}[commandchars=\\\{\}]
<IPython.core.display.HTML object>
    \end{Verbatim}

    
    
    \begin{Verbatim}[commandchars=\\\{\}]
<IPython.core.display.HTML object>
    \end{Verbatim}

    
    \begin{tcolorbox}[breakable, size=fbox, boxrule=1pt, pad at break*=1mm,colback=cellbackground, colframe=cellborder]
\prompt{In}{incolor}{14}{\boxspacing}
\begin{Verbatim}[commandchars=\\\{\}]
\PY{n}{vgg16\PYZus{}model} \PY{o}{=} \PY{n}{cnn\PYZus{}learner}\PY{p}{(}\PY{n}{dls}\PY{p}{,} \PY{n}{vgg16\PYZus{}bn}\PY{p}{,} \PY{n}{metrics}\PY{o}{=}\PY{n}{error\PYZus{}rate}\PY{p}{)}
\PY{n}{vgg16\PYZus{}model}\PY{o}{.}\PY{n}{fine\PYZus{}tune}\PY{p}{(}\PY{l+m+mi}{1}\PY{p}{)}
\end{Verbatim}
\end{tcolorbox}

    \begin{Verbatim}[commandchars=\\\{\}]
Downloading: "https://download.pytorch.org/models/vgg16\_bn-6c64b313.pth" to
/root/.cache/torch/hub/checkpoints/vgg16\_bn-6c64b313.pth
    \end{Verbatim}

    
    \begin{Verbatim}[commandchars=\\\{\}]
  0\%|          | 0.00/528M [00:00<?, ?B/s]
    \end{Verbatim}

    
    
    \begin{Verbatim}[commandchars=\\\{\}]
<IPython.core.display.HTML object>
    \end{Verbatim}

    
    
    \begin{Verbatim}[commandchars=\\\{\}]
<IPython.core.display.HTML object>
    \end{Verbatim}

    
    \begin{tcolorbox}[breakable, size=fbox, boxrule=1pt, pad at break*=1mm,colback=cellbackground, colframe=cellborder]
\prompt{In}{incolor}{15}{\boxspacing}
\begin{Verbatim}[commandchars=\\\{\}]
\PY{n}{resnet50\PYZus{}model} \PY{o}{=} \PY{n}{cnn\PYZus{}learner}\PY{p}{(}\PY{n}{dls}\PY{p}{,} \PY{n}{resnet50}\PY{p}{,} \PY{n}{metrics}\PY{o}{=}\PY{n}{error\PYZus{}rate}\PY{p}{)}
\PY{n}{resnet50\PYZus{}model}\PY{o}{.}\PY{n}{fine\PYZus{}tune}\PY{p}{(}\PY{l+m+mi}{1}\PY{p}{)}
\end{Verbatim}
\end{tcolorbox}

    \begin{Verbatim}[commandchars=\\\{\}]
Downloading: "https://download.pytorch.org/models/resnet50-0676ba61.pth" to
/root/.cache/torch/hub/checkpoints/resnet50-0676ba61.pth
    \end{Verbatim}

    
    \begin{Verbatim}[commandchars=\\\{\}]
  0\%|          | 0.00/97.8M [00:00<?, ?B/s]
    \end{Verbatim}

    
    
    \begin{Verbatim}[commandchars=\\\{\}]
<IPython.core.display.HTML object>
    \end{Verbatim}

    
    
    \begin{Verbatim}[commandchars=\\\{\}]
<IPython.core.display.HTML object>
    \end{Verbatim}

    
    \begin{tcolorbox}[breakable, size=fbox, boxrule=1pt, pad at break*=1mm,colback=cellbackground, colframe=cellborder]
\prompt{In}{incolor}{16}{\boxspacing}
\begin{Verbatim}[commandchars=\\\{\}]
\PY{n}{interp} \PY{o}{=} \PY{n}{ClassificationInterpretation}\PY{o}{.}\PY{n}{from\PYZus{}learner}\PY{p}{(}\PY{n}{resnet50\PYZus{}model}\PY{p}{)}
\PY{n}{interp}\PY{o}{.}\PY{n}{plot\PYZus{}confusion\PYZus{}matrix}\PY{p}{(}\PY{p}{)}
\end{Verbatim}
\end{tcolorbox}

    
    \begin{Verbatim}[commandchars=\\\{\}]
<IPython.core.display.HTML object>
    \end{Verbatim}

    
    \begin{center}
    \adjustimage{max size={0.9\linewidth}{0.9\paperheight}}{Project_Big_Data_files/Project_Big_Data_43_1.png}
    \end{center}
    { \hspace*{\fill} \\}
    
    \begin{tcolorbox}[breakable, size=fbox, boxrule=1pt, pad at break*=1mm,colback=cellbackground, colframe=cellborder]
\prompt{In}{incolor}{17}{\boxspacing}
\begin{Verbatim}[commandchars=\\\{\}]
\PY{n}{interp}\PY{o}{.}\PY{n}{plot\PYZus{}confusion\PYZus{}matrix}\PY{p}{(}\PY{n}{figsize}\PY{o}{=}\PY{p}{(}\PY{l+m+mi}{12}\PY{p}{,}\PY{l+m+mi}{12}\PY{p}{)}\PY{p}{,} \PY{n}{dpi}\PY{o}{=}\PY{l+m+mi}{60}\PY{p}{)}
\end{Verbatim}
\end{tcolorbox}

    \begin{center}
    \adjustimage{max size={0.9\linewidth}{0.9\paperheight}}{Project_Big_Data_files/Project_Big_Data_44_0.png}
    \end{center}
    { \hspace*{\fill} \\}
    
    \begin{tcolorbox}[breakable, size=fbox, boxrule=1pt, pad at break*=1mm,colback=cellbackground, colframe=cellborder]
\prompt{In}{incolor}{18}{\boxspacing}
\begin{Verbatim}[commandchars=\\\{\}]
\PY{n}{interp}\PY{o}{.}\PY{n}{plot\PYZus{}top\PYZus{}losses}\PY{p}{(}\PY{l+m+mi}{5}\PY{p}{,} \PY{n}{nrows}\PY{o}{=}\PY{l+m+mi}{1}\PY{p}{)}
\end{Verbatim}
\end{tcolorbox}

    \begin{center}
    \adjustimage{max size={0.9\linewidth}{0.9\paperheight}}{Project_Big_Data_files/Project_Big_Data_45_0.png}
    \end{center}
    { \hspace*{\fill} \\}
    
    \begin{tcolorbox}[breakable, size=fbox, boxrule=1pt, pad at break*=1mm,colback=cellbackground, colframe=cellborder]
\prompt{In}{incolor}{19}{\boxspacing}
\begin{Verbatim}[commandchars=\\\{\}]
\PY{n}{interp}\PY{o}{.}\PY{n}{most\PYZus{}confused}\PY{p}{(}\PY{n}{min\PYZus{}val}\PY{o}{=}\PY{l+m+mi}{5}\PY{p}{)}
\end{Verbatim}
\end{tcolorbox}

            \begin{tcolorbox}[breakable, size=fbox, boxrule=.5pt, pad at break*=1mm, opacityfill=0]
\prompt{Out}{outcolor}{19}{\boxspacing}
\begin{Verbatim}[commandchars=\\\{\}]
[('baroque', 'rococo', 25),
 ('expressionist', 'cunstructivist', 25),
 ('rococo', 'baroque', 25),
 ('cunstructivist', 'modern', 24),
 ('expressionist', 'modern', 24),
 ('brutalism', 'modern', 19),
 ('modern', 'brutalism', 16),
 ('federal', 'modern', 15),
 ('moorisch', 'indoislamic', 14),
 ('federal', 'rococo', 13),
 ('art deco', 'cunstructivist', 12),
 ('modern', 'federal', 12),
 ('moorisch', 'roman', 12),
 ('cunstructivist', 'expressionist', 11),
 ('cunstructivist', 'federal', 11),
 ('modern', 'cunstructivist', 11),
 ('modern', 'expressionist', 11),
 ('art deco', 'tudor', 10),
 ('federal', 'art deco', 10),
 ('federal', 'cunstructivist', 10),
 ('indoislamic', 'roman', 10),
 ('baroque', 'gothic', 9),
 ('cunstructivist', 'brutalism', 9),
 ('expressionist', 'art deco', 9),
 ('expressionist', 'brutalism', 9),
 ('expressionist', 'gothic', 9),
 ('roman', 'indoislamic', 9),
 ('brutalism', 'expressionist', 8),
 ('cunstructivist', 'art deco', 8),
 ('federal', 'tudor', 8),
 ('gothic', 'indoislamic', 8),
 ('tudor', 'federal', 8),
 ('ancient egyptian', 'roman', 7),
 ('art deco', 'federal', 7),
 ('art deco', 'modern', 7),
 ('expressionist', 'baroque', 7),
 ('expressionist', 'indoislamic', 7),
 ('federal', 'brutalism', 7),
 ('federal', 'expressionist', 7),
 ('indoislamic', 'cunstructivist', 7),
 ('modern', 'tudor', 7),
 ('tudor', 'gothic', 7),
 ('art deco', 'expressionist', 6),
 ('cunstructivist', 'indoislamic', 6),
 ('cunstructivist', 'tudor', 6),
 ('expressionist', 'federal', 6),
 ('federal', 'indoislamic', 6),
 ('indoislamic', 'moorisch', 6),
 ('modern', 'art deco', 6),
 ('moorisch', 'ancient egyptian', 6),
 ('ancient egyptian', 'indoislamic', 5),
 ('art deco', 'indoislamic', 5),
 ('art deco', 'moorisch', 5),
 ('brutalism', 'cunstructivist', 5),
 ('federal', 'moorisch', 5),
 ('indoislamic', 'ancient egyptian', 5),
 ('indoislamic', 'expressionist', 5),
 ('roman', 'ancient egyptian', 5),
 ('tudor', 'moorisch', 5)]
\end{Verbatim}
\end{tcolorbox}
        
    \begin{tcolorbox}[breakable, size=fbox, boxrule=1pt, pad at break*=1mm,colback=cellbackground, colframe=cellborder]
\prompt{In}{incolor}{20}{\boxspacing}
\begin{Verbatim}[commandchars=\\\{\}]
\PY{n}{resnet34\PYZus{}overtrain} \PY{o}{=} \PY{n}{cnn\PYZus{}learner}\PY{p}{(}\PY{n}{dls}\PY{p}{,} \PY{n}{resnet34}\PY{p}{,} \PY{n}{metrics}\PY{o}{=}\PY{n}{error\PYZus{}rate}\PY{p}{)}
\PY{n}{resnet34\PYZus{}overtrain}\PY{o}{.}\PY{n}{fine\PYZus{}tune}\PY{p}{(}\PY{l+m+mi}{1}\PY{p}{,} \PY{n}{base\PYZus{}lr}\PY{o}{=}\PY{l+m+mf}{0.1}\PY{p}{)}
\end{Verbatim}
\end{tcolorbox}

    \begin{Verbatim}[commandchars=\\\{\}]
Downloading: "https://download.pytorch.org/models/resnet34-b627a593.pth" to
/root/.cache/torch/hub/checkpoints/resnet34-b627a593.pth
    \end{Verbatim}

    
    \begin{Verbatim}[commandchars=\\\{\}]
  0\%|          | 0.00/83.3M [00:00<?, ?B/s]
    \end{Verbatim}

    
    
    \begin{Verbatim}[commandchars=\\\{\}]
<IPython.core.display.HTML object>
    \end{Verbatim}

    
    
    \begin{Verbatim}[commandchars=\\\{\}]
<IPython.core.display.HTML object>
    \end{Verbatim}

    
    \begin{tcolorbox}[breakable, size=fbox, boxrule=1pt, pad at break*=1mm,colback=cellbackground, colframe=cellborder]
\prompt{In}{incolor}{21}{\boxspacing}
\begin{Verbatim}[commandchars=\\\{\}]
\PY{n}{resnet34\PYZus{}just\PYZus{}right} \PY{o}{=} \PY{n}{cnn\PYZus{}learner}\PY{p}{(}\PY{n}{dls}\PY{p}{,} \PY{n}{resnet34}\PY{p}{,} \PY{n}{metrics}\PY{o}{=}\PY{n}{error\PYZus{}rate}\PY{p}{)}
\PY{n}{lr\PYZus{}min}\PY{p}{,}\PY{n}{lr\PYZus{}steep} \PY{o}{=} \PY{n}{resnet34\PYZus{}just\PYZus{}right}\PY{o}{.}\PY{n}{lr\PYZus{}find}\PY{p}{(}\PY{p}{)}
\end{Verbatim}
\end{tcolorbox}

    
    \begin{Verbatim}[commandchars=\\\{\}]
<IPython.core.display.HTML object>
    \end{Verbatim}

    
    \begin{Verbatim}[commandchars=\\\{\}, frame=single, framerule=2mm, rulecolor=\color{outerrorbackground}]
\textcolor{ansi-red}{---------------------------------------------------------------------------}
\textcolor{ansi-red}{ValueError}                                Traceback (most recent call last)
\textcolor{ansi-green}{<ipython-input-21-182f03c43633>} in \textcolor{ansi-cyan}{<module>}\textcolor{ansi-blue}{()}
\textcolor{ansi-green-intense}{\textbf{      1}} resnet34\_just\_right \textcolor{ansi-blue}{=} cnn\_learner\textcolor{ansi-blue}{(}dls\textcolor{ansi-blue}{,} resnet34\textcolor{ansi-blue}{,} metrics\textcolor{ansi-blue}{=}error\_rate\textcolor{ansi-blue}{)}
\textcolor{ansi-green}{----> 2}\textcolor{ansi-red}{ }lr\_min\textcolor{ansi-blue}{,}lr\_steep \textcolor{ansi-blue}{=} resnet34\_just\_right\textcolor{ansi-blue}{.}lr\_find\textcolor{ansi-blue}{(}\textcolor{ansi-blue}{)}

\textcolor{ansi-red}{ValueError}: not enough values to unpack (expected 2, got 1)
    \end{Verbatim}

    \begin{center}
    \adjustimage{max size={0.9\linewidth}{0.9\paperheight}}{Project_Big_Data_files/Project_Big_Data_48_2.png}
    \end{center}
    { \hspace*{\fill} \\}
    
    \begin{tcolorbox}[breakable, size=fbox, boxrule=1pt, pad at break*=1mm,colback=cellbackground, colframe=cellborder]
\prompt{In}{incolor}{22}{\boxspacing}
\begin{Verbatim}[commandchars=\\\{\}]
\PY{n+nb}{print}\PY{p}{(}\PY{l+s+sa}{f}\PY{l+s+s2}{\PYZdq{}}\PY{l+s+s2}{Steepest point: }\PY{l+s+si}{\PYZob{}}\PY{n}{lr\PYZus{}steep}\PY{l+s+si}{:}\PY{l+s+s2}{.2e}\PY{l+s+si}{\PYZcb{}}\PY{l+s+s2}{\PYZdq{}}\PY{p}{)}
\end{Verbatim}
\end{tcolorbox}

    \begin{Verbatim}[commandchars=\\\{\}, frame=single, framerule=2mm, rulecolor=\color{outerrorbackground}]
\textcolor{ansi-red}{---------------------------------------------------------------------------}
\textcolor{ansi-red}{NameError}                                 Traceback (most recent call last)
\textcolor{ansi-green}{<ipython-input-22-c845ce598eda>} in \textcolor{ansi-cyan}{<module>}\textcolor{ansi-blue}{()}
\textcolor{ansi-green}{----> 1}\textcolor{ansi-red}{ }print\textcolor{ansi-blue}{(}\textcolor{ansi-blue}{f"Steepest point: \{lr\_steep:.2e\}"}\textcolor{ansi-blue}{)}

\textcolor{ansi-red}{NameError}: name 'lr\_steep' is not defined
    \end{Verbatim}

    \begin{tcolorbox}[breakable, size=fbox, boxrule=1pt, pad at break*=1mm,colback=cellbackground, colframe=cellborder]
\prompt{In}{incolor}{23}{\boxspacing}
\begin{Verbatim}[commandchars=\\\{\}]
\PY{n}{resnet\PYZus{}adv} \PY{o}{=} \PY{n}{cnn\PYZus{}learner}\PY{p}{(}\PY{n}{dls}\PY{p}{,} \PY{n}{resnet34}\PY{p}{,} \PY{n}{metrics}\PY{o}{=}\PY{n}{error\PYZus{}rate}\PY{p}{)}
\PY{n}{resnet\PYZus{}adv}\PY{o}{.}\PY{n}{fit\PYZus{}one\PYZus{}cycle}\PY{p}{(}\PY{l+m+mi}{3}\PY{p}{,} \PY{l+m+mf}{3e\PYZhy{}3}\PY{p}{)}
\end{Verbatim}
\end{tcolorbox}

    
    \begin{Verbatim}[commandchars=\\\{\}]
<IPython.core.display.HTML object>
    \end{Verbatim}

    
    \begin{tcolorbox}[breakable, size=fbox, boxrule=1pt, pad at break*=1mm,colback=cellbackground, colframe=cellborder]
\prompt{In}{incolor}{24}{\boxspacing}
\begin{Verbatim}[commandchars=\\\{\}]
\PY{n}{resnet\PYZus{}adv}\PY{o}{.}\PY{n}{unfreeze}\PY{p}{(}\PY{p}{)}
\end{Verbatim}
\end{tcolorbox}

    \begin{tcolorbox}[breakable, size=fbox, boxrule=1pt, pad at break*=1mm,colback=cellbackground, colframe=cellborder]
\prompt{In}{incolor}{25}{\boxspacing}
\begin{Verbatim}[commandchars=\\\{\}]
\PY{n}{resnet\PYZus{}adv}\PY{o}{.}\PY{n}{lr\PYZus{}find}\PY{p}{(}\PY{p}{)}
\end{Verbatim}
\end{tcolorbox}

    
    \begin{Verbatim}[commandchars=\\\{\}]
<IPython.core.display.HTML object>
    \end{Verbatim}

    
            \begin{tcolorbox}[breakable, size=fbox, boxrule=.5pt, pad at break*=1mm, opacityfill=0]
\prompt{Out}{outcolor}{25}{\boxspacing}
\begin{Verbatim}[commandchars=\\\{\}]
SuggestedLRs(valley=6.918309736647643e-06)
\end{Verbatim}
\end{tcolorbox}
        
    \begin{center}
    \adjustimage{max size={0.9\linewidth}{0.9\paperheight}}{Project_Big_Data_files/Project_Big_Data_52_2.png}
    \end{center}
    { \hspace*{\fill} \\}
    
    \begin{tcolorbox}[breakable, size=fbox, boxrule=1pt, pad at break*=1mm,colback=cellbackground, colframe=cellborder]
\prompt{In}{incolor}{26}{\boxspacing}
\begin{Verbatim}[commandchars=\\\{\}]
\PY{n}{resnet\PYZus{}adv}\PY{o}{.}\PY{n}{fit\PYZus{}one\PYZus{}cycle}\PY{p}{(}\PY{l+m+mi}{6}\PY{p}{,} \PY{n}{lr\PYZus{}max}\PY{o}{=}\PY{l+m+mf}{1e\PYZhy{}5}\PY{p}{)}
\end{Verbatim}
\end{tcolorbox}

    
    \begin{Verbatim}[commandchars=\\\{\}]
<IPython.core.display.HTML object>
    \end{Verbatim}

    
    \begin{tcolorbox}[breakable, size=fbox, boxrule=1pt, pad at break*=1mm,colback=cellbackground, colframe=cellborder]
\prompt{In}{incolor}{27}{\boxspacing}
\begin{Verbatim}[commandchars=\\\{\}]
\PY{n}{resnet\PYZus{}adv}\PY{o}{.}\PY{n}{recorder}\PY{o}{.}\PY{n}{plot\PYZus{}loss}\PY{p}{(}\PY{p}{)}
\end{Verbatim}
\end{tcolorbox}

    \begin{center}
    \adjustimage{max size={0.9\linewidth}{0.9\paperheight}}{Project_Big_Data_files/Project_Big_Data_54_0.png}
    \end{center}
    { \hspace*{\fill} \\}
    
    \begin{tcolorbox}[breakable, size=fbox, boxrule=1pt, pad at break*=1mm,colback=cellbackground, colframe=cellborder]
\prompt{In}{incolor}{28}{\boxspacing}
\begin{Verbatim}[commandchars=\\\{\}]
\PY{n}{resnet\PYZus{}really\PYZus{}adv} \PY{o}{=} \PY{n}{cnn\PYZus{}learner}\PY{p}{(}\PY{n}{dls}\PY{p}{,} \PY{n}{resnet34}\PY{p}{,} \PY{n}{metrics}\PY{o}{=}\PY{n}{error\PYZus{}rate}\PY{p}{)}
\PY{n}{resnet\PYZus{}really\PYZus{}adv}\PY{o}{.}\PY{n}{fit\PYZus{}one\PYZus{}cycle}\PY{p}{(}\PY{l+m+mi}{3}\PY{p}{,} \PY{l+m+mf}{3e\PYZhy{}3}\PY{p}{)}
\PY{n}{resnet\PYZus{}really\PYZus{}adv}\PY{o}{.}\PY{n}{unfreeze}\PY{p}{(}\PY{p}{)}
\PY{n}{resnet\PYZus{}really\PYZus{}adv}\PY{o}{.}\PY{n}{fit\PYZus{}one\PYZus{}cycle}\PY{p}{(}\PY{l+m+mi}{12}\PY{p}{,} \PY{n}{lr\PYZus{}max}\PY{o}{=}\PY{n+nb}{slice}\PY{p}{(}\PY{l+m+mf}{1e\PYZhy{}6}\PY{p}{,}\PY{l+m+mf}{1e\PYZhy{}4}\PY{p}{)}\PY{p}{)}
\end{Verbatim}
\end{tcolorbox}

    
    \begin{Verbatim}[commandchars=\\\{\}]
<IPython.core.display.HTML object>
    \end{Verbatim}

    
    
    \begin{Verbatim}[commandchars=\\\{\}]
<IPython.core.display.HTML object>
    \end{Verbatim}

    
    \begin{tcolorbox}[breakable, size=fbox, boxrule=1pt, pad at break*=1mm,colback=cellbackground, colframe=cellborder]
\prompt{In}{incolor}{29}{\boxspacing}
\begin{Verbatim}[commandchars=\\\{\}]
\PY{n}{resnet\PYZus{}really\PYZus{}adv}\PY{o}{.}\PY{n}{recorder}\PY{o}{.}\PY{n}{plot\PYZus{}loss}\PY{p}{(}\PY{p}{)}
\end{Verbatim}
\end{tcolorbox}

    \begin{center}
    \adjustimage{max size={0.9\linewidth}{0.9\paperheight}}{Project_Big_Data_files/Project_Big_Data_56_0.png}
    \end{center}
    { \hspace*{\fill} \\}
    
    \begin{tcolorbox}[breakable, size=fbox, boxrule=1pt, pad at break*=1mm,colback=cellbackground, colframe=cellborder]
\prompt{In}{incolor}{30}{\boxspacing}
\begin{Verbatim}[commandchars=\\\{\}]
\PY{k+kn}{from} \PY{n+nn}{fastai}\PY{n+nn}{.}\PY{n+nn}{callback}\PY{n+nn}{.}\PY{n+nn}{fp16} \PY{k+kn}{import} \PY{o}{*}
\PY{n}{resnet\PYZus{}adv\PYZus{}tweaked} \PY{o}{=} \PY{n}{cnn\PYZus{}learner}\PY{p}{(}\PY{n}{dls}\PY{p}{,} \PY{n}{resnet50}\PY{p}{,} \PY{n}{metrics}\PY{o}{=}\PY{n}{error\PYZus{}rate}\PY{p}{)}\PY{o}{.}\PY{n}{to\PYZus{}fp16}\PY{p}{(}\PY{p}{)}
\PY{n}{resnet\PYZus{}adv\PYZus{}tweaked}\PY{o}{.}\PY{n}{fine\PYZus{}tune}\PY{p}{(}\PY{l+m+mi}{6}\PY{p}{,} \PY{n}{freeze\PYZus{}epochs}\PY{o}{=}\PY{l+m+mi}{3}\PY{p}{)}
\end{Verbatim}
\end{tcolorbox}

    
    \begin{Verbatim}[commandchars=\\\{\}]
<IPython.core.display.HTML object>
    \end{Verbatim}

    
    
    \begin{Verbatim}[commandchars=\\\{\}]
<IPython.core.display.HTML object>
    \end{Verbatim}

    
    \begin{tcolorbox}[breakable, size=fbox, boxrule=1pt, pad at break*=1mm,colback=cellbackground, colframe=cellborder]
\prompt{In}{incolor}{31}{\boxspacing}
\begin{Verbatim}[commandchars=\\\{\}]
\PY{n}{resnet\PYZus{}adv\PYZus{}tweaked}\PY{o}{.}\PY{n}{recorder}\PY{o}{.}\PY{n}{plot\PYZus{}loss}\PY{p}{(}\PY{p}{)}
\end{Verbatim}
\end{tcolorbox}

    \begin{center}
    \adjustimage{max size={0.9\linewidth}{0.9\paperheight}}{Project_Big_Data_files/Project_Big_Data_58_0.png}
    \end{center}
    { \hspace*{\fill} \\}
    
    \begin{tcolorbox}[breakable, size=fbox, boxrule=1pt, pad at break*=1mm,colback=cellbackground, colframe=cellborder]
\prompt{In}{incolor}{33}{\boxspacing}
\begin{Verbatim}[commandchars=\\\{\}]
\PY{n}{resnet\PYZus{}really\PYZus{}adv}\PY{o}{.}\PY{n}{export}\PY{p}{(}\PY{p}{)}
\end{Verbatim}
\end{tcolorbox}


    
    
\newpage
    

    
\section{Flask}

Dependencies

    \begin{tcolorbox}[breakable, size=fbox, boxrule=1pt, pad at break*=1mm,colback=cellbackground, colframe=cellborder]
\prompt{In}{incolor}{14}{\boxspacing}
\begin{Verbatim}[commandchars=\\\{\}]
\PY{o}{!}pip install fastai
\PY{o}{!}pip install flask
\end{Verbatim}
\end{tcolorbox}


    Imports

The imports for the flask website. We used os to make directories and re
to do some regex on the uploaded files.

There was also a dependency issue with load\_learner from
fastai.vision.all, tt used PosixPath but that isn't used in windows so
we fixed it by replacing PosixPath with the WindowsPath.

    \begin{tcolorbox}[breakable, size=fbox, boxrule=1pt, pad at break*=1mm,colback=cellbackground, colframe=cellborder]
\prompt{In}{incolor}{17}{\boxspacing}
\begin{Verbatim}[commandchars=\\\{\}]
\PY{k+kn}{from} \PY{n+nn}{flask} \PY{k+kn}{import} \PY{n}{Flask}\PY{p}{,} \PY{n}{render\PYZus{}template}\PY{p}{,} \PY{n}{request}\PY{p}{,} \PY{n}{redirect}\PY{p}{,} \PY{n}{url\PYZus{}for}
\PY{k+kn}{from} \PY{n+nn}{werkzeug}\PY{n+nn}{.}\PY{n+nn}{utils} \PY{k+kn}{import} \PY{n}{secure\PYZus{}filename}
\PY{k+kn}{import} \PY{n+nn}{os}
\PY{k+kn}{import} \PY{n+nn}{re}


\PY{k+kn}{import} \PY{n+nn}{random} \PY{k}{as} \PY{n+nn}{rnd}
\PY{k+kn}{from} \PY{n+nn}{fastai} \PY{k+kn}{import} \PY{o}{*}
\PY{k+kn}{from} \PY{n+nn}{fastai}\PY{n+nn}{.}\PY{n+nn}{vision}\PY{n+nn}{.}\PY{n+nn}{all} \PY{k+kn}{import} \PY{o}{*}

\PY{c+c1}{\PYZsh{} fixes a dependancy on PosixPath by load\PYZus{}learner}
\PY{k+kn}{import} \PY{n+nn}{pathlib}
\PY{n}{temp} \PY{o}{=} \PY{n}{pathlib}\PY{o}{.}\PY{n}{PosixPath}
\PY{n}{pathlib}\PY{o}{.}\PY{n}{PosixPath} \PY{o}{=} \PY{n}{pathlib}\PY{o}{.}\PY{n}{WindowsPath}
\end{Verbatim}
\end{tcolorbox}

    Global variables

These contain fixed data that is used by the website. The image that is
uploaded on the website to be tested by the model is saved in the
uploadfolder.

    \begin{tcolorbox}[breakable, size=fbox, boxrule=1pt, pad at break*=1mm,colback=cellbackground, colframe=cellborder]
\prompt{In}{incolor}{18}{\boxspacing}
\begin{Verbatim}[commandchars=\\\{\}]
\PY{n}{app} \PY{o}{=} \PY{n}{Flask}\PY{p}{(}\PY{n+nv+vm}{\PYZus{}\PYZus{}name\PYZus{}\PYZus{}}\PY{p}{)}
\PY{n}{app}\PY{o}{.}\PY{n}{config}\PY{p}{[}\PY{l+s+s1}{\PYZsq{}}\PY{l+s+s1}{UPLOAD\PYZus{}FOLDER}\PY{l+s+s1}{\PYZsq{}}\PY{p}{]} \PY{o}{=} \PY{l+s+s1}{\PYZsq{}}\PY{l+s+s1}{./static/UploadFolder}\PY{l+s+s1}{\PYZsq{}}

\PY{n}{DATAPATH} \PY{o}{=} \PY{l+s+s2}{\PYZdq{}}\PY{l+s+s2}{./static/data/}\PY{l+s+s2}{\PYZdq{}}
\PY{c+c1}{\PYZsh{} These are the categories used in the model and the data}
\PY{n}{CATEGORIES} \PY{o}{=} \PY{n}{os}\PY{o}{.}\PY{n}{listdir}\PY{p}{(}\PY{n}{DATAPATH}\PY{p}{)}
\PY{n}{ALLOWED\PYZus{}IMGTYPES} \PY{o}{=} \PY{p}{[}\PY{l+s+s1}{\PYZsq{}}\PY{l+s+s1}{png}\PY{l+s+s1}{\PYZsq{}}\PY{p}{,}\PY{l+s+s1}{\PYZsq{}}\PY{l+s+s1}{jpeg}\PY{l+s+s1}{\PYZsq{}}\PY{p}{,}\PY{l+s+s1}{\PYZsq{}}\PY{l+s+s1}{gif}\PY{l+s+s1}{\PYZsq{}}\PY{p}{,}\PY{l+s+s1}{\PYZsq{}}\PY{l+s+s1}{jfif}\PY{l+s+s1}{\PYZsq{}}\PY{p}{,}\PY{l+s+s1}{\PYZsq{}}\PY{l+s+s1}{jpg}\PY{l+s+s1}{\PYZsq{}}\PY{p}{]}
\PY{c+c1}{\PYZsh{} uses top 20 images from the scrapers}
\PY{n}{N\PYZus{}MAX} \PY{o}{=} \PY{l+m+mi}{20}

\PY{c+c1}{\PYZsh{} learn is the model that is loaded from the pkl file. this file contains a model that predicts 14 different archtectal styles}
\PY{n}{learn} \PY{o}{=} \PY{n}{load\PYZus{}learner}\PY{p}{(}\PY{l+s+s1}{\PYZsq{}}\PY{l+s+s1}{exportBig.pkl}\PY{l+s+s1}{\PYZsq{}}\PY{p}{)}
\end{Verbatim}
\end{tcolorbox}

    Format functions

These functions are used to format the data that has to be represented
in the webpage. Example is a `card' containing the right category, the
image and the predicted category.

Sample is the formatting of the uploaded image to be tested.

    \begin{tcolorbox}[breakable, size=fbox, boxrule=1pt, pad at break*=1mm,colback=cellbackground, colframe=cellborder]
\prompt{In}{incolor}{19}{\boxspacing}
\begin{Verbatim}[commandchars=\\\{\}]
\PY{k}{def} \PY{n+nf}{format\PYZus{}example}\PY{p}{(}\PY{n}{imgpath}\PY{p}{,}\PY{n}{cat}\PY{p}{,}\PY{n}{prediction}\PY{p}{)}\PY{p}{:}
   \PY{k}{return} \PY{p}{\PYZob{}}\PY{l+s+s2}{\PYZdq{}}\PY{l+s+s2}{image}\PY{l+s+s2}{\PYZdq{}}\PY{p}{:}\PY{l+s+s2}{\PYZdq{}}\PY{l+s+s2}{/data/}\PY{l+s+s2}{\PYZdq{}}\PY{o}{+}\PY{n}{imgpath}\PY{p}{,}
                           \PY{l+s+s2}{\PYZdq{}}\PY{l+s+s2}{category}\PY{l+s+s2}{\PYZdq{}}\PY{p}{:}\PY{n}{cat}\PY{p}{,}
                           \PY{l+s+s2}{\PYZdq{}}\PY{l+s+s2}{prediction}\PY{l+s+s2}{\PYZdq{}}\PY{p}{:}\PY{n}{prediction}\PY{p}{[}\PY{l+m+mi}{0}\PY{p}{]}\PY{p}{,}
                           \PY{l+s+s2}{\PYZdq{}}\PY{l+s+s2}{correct\PYZus{}class}\PY{l+s+s2}{\PYZdq{}}\PY{p}{:}\PY{l+s+s2}{\PYZdq{}}\PY{l+s+s2}{example\PYZus{}}\PY{l+s+s2}{\PYZdq{}}\PY{o}{+}\PY{n+nb}{str}\PY{p}{(}\PY{n}{cat}\PY{o}{==}\PY{n}{prediction}\PY{p}{[}\PY{l+m+mi}{0}\PY{p}{]}\PY{p}{)}\PY{p}{\PYZcb{}}

\PY{k}{def} \PY{n+nf}{format\PYZus{}sample}\PY{p}{(}\PY{n}{imgpath}\PY{p}{,}\PY{n}{prediction}\PY{p}{)}\PY{p}{:}
   \PY{k}{return} \PY{p}{\PYZob{}}\PY{l+s+s2}{\PYZdq{}}\PY{l+s+s2}{image}\PY{l+s+s2}{\PYZdq{}}\PY{p}{:}\PY{l+s+s2}{\PYZdq{}}\PY{l+s+s2}{/UploadFolder/}\PY{l+s+s2}{\PYZdq{}}\PY{o}{+}\PY{n}{imgpath}\PY{p}{,}
            \PY{l+s+s2}{\PYZdq{}}\PY{l+s+s2}{prediction}\PY{l+s+s2}{\PYZdq{}}\PY{p}{:}\PY{n}{prediction}\PY{p}{[}\PY{l+m+mi}{0}\PY{p}{]}\PY{p}{,}
            \PY{l+s+s2}{\PYZdq{}}\PY{l+s+s2}{sample\PYZus{}class}\PY{l+s+s2}{\PYZdq{}}\PY{p}{:}\PY{l+s+s2}{\PYZdq{}}\PY{l+s+s2}{sample\PYZus{}visable}\PY{l+s+s2}{\PYZdq{}}\PY{p}{\PYZcb{}}
\end{Verbatim}
\end{tcolorbox}

    Examples function

This function gets a random image from every category and returns a list
of all the formatted examples.

    \begin{tcolorbox}[breakable, size=fbox, boxrule=1pt, pad at break*=1mm,colback=cellbackground, colframe=cellborder]
\prompt{In}{incolor}{20}{\boxspacing}
\begin{Verbatim}[commandchars=\\\{\}]
\PY{k}{def} \PY{n+nf}{get\PYZus{}examples}\PY{p}{(}\PY{p}{)}\PY{p}{:}
   \PY{n}{examples} \PY{o}{=} \PY{p}{[}\PY{p}{]}
   \PY{k}{for} \PY{n}{cat} \PY{o+ow}{in} \PY{n}{CATEGORIES}\PY{p}{:}
      \PY{n}{n} \PY{o}{=} \PY{n}{rnd}\PY{o}{.}\PY{n}{choice}\PY{p}{(}\PY{n+nb}{range}\PY{p}{(}\PY{n}{N\PYZus{}MAX}\PY{p}{)}\PY{p}{)}
      \PY{n}{imgpath} \PY{o}{=} \PY{n}{cat}\PY{o}{+}\PY{l+s+s2}{\PYZdq{}}\PY{l+s+s2}{/}\PY{l+s+s2}{\PYZdq{}}\PY{o}{+}\PY{n}{cat}\PY{o}{+}\PY{n+nb}{str}\PY{p}{(}\PY{n}{n}\PY{p}{)}\PY{o}{+}\PY{l+s+s2}{\PYZdq{}}\PY{l+s+s2}{.png}\PY{l+s+s2}{\PYZdq{}}
      \PY{n}{prediction} \PY{o}{=} \PY{n}{learn}\PY{o}{.}\PY{n}{predict}\PY{p}{(}\PY{n}{DATAPATH}\PY{o}{+}\PY{n}{imgpath}\PY{p}{)}

      \PY{n}{examples}\PY{o}{.}\PY{n}{append}\PY{p}{(}\PY{n}{format\PYZus{}example}\PY{p}{(}\PY{n}{imgpath}\PY{p}{,}\PY{n}{cat}\PY{p}{,}\PY{n}{prediction}\PY{p}{)}\PY{p}{)}
   
   \PY{k}{return} \PY{n}{examples}
\end{Verbatim}
\end{tcolorbox}

    Flask routed functions

These functions are routed by Flask and return the same template (home).
The difference is that the initial routing is used when no image is
uploaded and the second function is used when an image is uploaded.

    \begin{tcolorbox}[breakable, size=fbox, boxrule=1pt, pad at break*=1mm,colback=cellbackground, colframe=cellborder]
\prompt{In}{incolor}{21}{\boxspacing}
\begin{Verbatim}[commandchars=\\\{\}]
\PY{n+nd}{@app}\PY{o}{.}\PY{n}{route}\PY{p}{(}\PY{l+s+s1}{\PYZsq{}}\PY{l+s+s1}{/}\PY{l+s+s1}{\PYZsq{}}\PY{p}{)}
\PY{k}{def} \PY{n+nf}{home}\PY{p}{(}\PY{p}{)}\PY{p}{:}
   \PY{n}{examples} \PY{o}{=} \PY{n}{get\PYZus{}examples}\PY{p}{(}\PY{p}{)}
   \PY{k}{return} \PY{n}{render\PYZus{}template}\PY{p}{(}\PY{l+s+s1}{\PYZsq{}}\PY{l+s+s1}{home.html}\PY{l+s+s1}{\PYZsq{}}\PY{p}{,}\PY{n}{examples}\PY{o}{=}\PY{n}{examples}\PY{p}{,}\PY{n}{sample}\PY{o}{=}\PY{p}{\PYZob{}}\PY{l+s+s2}{\PYZdq{}}\PY{l+s+s2}{sample\PYZus{}class}\PY{l+s+s2}{\PYZdq{}}\PY{p}{:}\PY{l+s+s2}{\PYZdq{}}\PY{l+s+s2}{sample\PYZus{}hidden}\PY{l+s+s2}{\PYZdq{}}\PY{p}{\PYZcb{}}\PY{p}{)}

\PY{n+nd}{@app}\PY{o}{.}\PY{n}{route}\PY{p}{(}\PY{l+s+s2}{\PYZdq{}}\PY{l+s+s2}{/testImage\PYZus{}post}\PY{l+s+s2}{\PYZdq{}}\PY{p}{,} \PY{n}{methods}\PY{o}{=}\PY{p}{[}\PY{l+s+s1}{\PYZsq{}}\PY{l+s+s1}{post}\PY{l+s+s1}{\PYZsq{}}\PY{p}{]}\PY{p}{)}
\PY{k}{def} \PY{n+nf}{testImage\PYZus{}post}\PY{p}{(}\PY{p}{)}\PY{p}{:}

   \PY{c+c1}{\PYZsh{} clears the upload folder, to prevent to much space being used}
   \PY{k}{for} \PY{n}{img} \PY{o+ow}{in} \PY{n}{os}\PY{o}{.}\PY{n}{listdir}\PY{p}{(}\PY{n}{app}\PY{o}{.}\PY{n}{config}\PY{p}{[}\PY{l+s+s1}{\PYZsq{}}\PY{l+s+s1}{UPLOAD\PYZus{}FOLDER}\PY{l+s+s1}{\PYZsq{}}\PY{p}{]}\PY{p}{)}\PY{p}{:}
      \PY{n}{os}\PY{o}{.}\PY{n}{remove}\PY{p}{(}\PY{n}{os}\PY{o}{.}\PY{n}{path}\PY{o}{.}\PY{n}{join}\PY{p}{(}\PY{n}{app}\PY{o}{.}\PY{n}{config}\PY{p}{[}\PY{l+s+s1}{\PYZsq{}}\PY{l+s+s1}{UPLOAD\PYZus{}FOLDER}\PY{l+s+s1}{\PYZsq{}}\PY{p}{]}\PY{p}{,}\PY{n}{img}\PY{p}{)}\PY{p}{)}

   \PY{c+c1}{\PYZsh{} requests file from the posted files, this is suposed to be an image}
   \PY{n}{image} \PY{o}{=} \PY{n}{request}\PY{o}{.}\PY{n}{files}\PY{p}{[}\PY{l+s+s1}{\PYZsq{}}\PY{l+s+s1}{file}\PY{l+s+s1}{\PYZsq{}}\PY{p}{]}
   \PY{n}{imgpath} \PY{o}{=} \PY{n}{secure\PYZus{}filename}\PY{p}{(}\PY{n}{image}\PY{o}{.}\PY{n}{filename}\PY{p}{)}

   \PY{c+c1}{\PYZsh{}regex function to see if uploaded file was an image, else return to home function}
   \PY{k}{if} \PY{n}{re}\PY{o}{.}\PY{n}{sub}\PY{p}{(}\PY{l+s+s1}{\PYZsq{}}\PY{l+s+s1}{.*}\PY{l+s+se}{\PYZbs{}\PYZbs{}}\PY{l+s+s1}{.}\PY{l+s+s1}{\PYZsq{}}\PY{p}{,}\PY{l+s+s1}{\PYZsq{}}\PY{l+s+s1}{\PYZsq{}}\PY{p}{,}\PY{n}{imgpath}\PY{p}{)} \PY{o+ow}{not} \PY{o+ow}{in} \PY{n}{ALLOWED\PYZus{}IMGTYPES}\PY{p}{:}
      \PY{k}{return} \PY{n}{redirect}\PY{p}{(}\PY{n}{url\PYZus{}for}\PY{p}{(}\PY{l+s+s1}{\PYZsq{}}\PY{l+s+s1}{home}\PY{l+s+s1}{\PYZsq{}}\PY{p}{)}\PY{p}{)}

   \PY{n}{image}\PY{o}{.}\PY{n}{save}\PY{p}{(}\PY{n}{os}\PY{o}{.}\PY{n}{path}\PY{o}{.}\PY{n}{join}\PY{p}{(}\PY{n}{app}\PY{o}{.}\PY{n}{config}\PY{p}{[}\PY{l+s+s1}{\PYZsq{}}\PY{l+s+s1}{UPLOAD\PYZus{}FOLDER}\PY{l+s+s1}{\PYZsq{}}\PY{p}{]}\PY{p}{,} \PY{n}{imgpath}\PY{p}{)}\PY{p}{)}

   \PY{c+c1}{\PYZsh{}test the posted image with the model}
   \PY{n}{prediction} \PY{o}{=} \PY{n}{learn}\PY{o}{.}\PY{n}{predict}\PY{p}{(}\PY{n}{os}\PY{o}{.}\PY{n}{path}\PY{o}{.}\PY{n}{join}\PY{p}{(}\PY{n}{app}\PY{o}{.}\PY{n}{config}\PY{p}{[}\PY{l+s+s1}{\PYZsq{}}\PY{l+s+s1}{UPLOAD\PYZus{}FOLDER}\PY{l+s+s1}{\PYZsq{}}\PY{p}{]}\PY{p}{,} \PY{n}{imgpath}\PY{p}{)}\PY{p}{)}

   \PY{n}{sample} \PY{o}{=} \PY{n}{format\PYZus{}sample}\PY{p}{(}\PY{n}{imgpath}\PY{p}{,}\PY{n}{prediction}\PY{p}{)}
   \PY{n}{examples} \PY{o}{=} \PY{n}{get\PYZus{}examples}\PY{p}{(}\PY{p}{)}
   \PY{k}{return} \PY{n}{render\PYZus{}template}\PY{p}{(}\PY{l+s+s1}{\PYZsq{}}\PY{l+s+s1}{home.html}\PY{l+s+s1}{\PYZsq{}}\PY{p}{,}\PY{n}{examples}\PY{o}{=}\PY{n}{examples}\PY{p}{,}\PY{n}{sample}\PY{o}{=}\PY{n}{sample}\PY{p}{)}
\end{Verbatim}
\end{tcolorbox}

    Run the server

    \begin{tcolorbox}[breakable, size=fbox, boxrule=1pt, pad at break*=1mm,colback=cellbackground, colframe=cellborder]
\prompt{In}{incolor}{22}{\boxspacing}
\begin{Verbatim}[commandchars=\\\{\}]
\PY{k}{if} \PY{n+nv+vm}{\PYZus{}\PYZus{}name\PYZus{}\PYZus{}} \PY{o}{==} \PY{l+s+s1}{\PYZsq{}}\PY{l+s+s1}{\PYZus{}\PYZus{}main\PYZus{}\PYZus{}}\PY{l+s+s1}{\PYZsq{}}\PY{p}{:}
   \PY{n}{app}\PY{o}{.}\PY{n}{run}\PY{p}{(}\PY{p}{)}
\end{Verbatim}
\end{tcolorbox}

    \begin{Verbatim}[commandchars=\\\{\}]
 * Serving Flask app '\_\_main\_\_' (lazy loading)
 * Environment: production
   WARNING: This is a development server. Do not use it in a production
deployment.
   Use a production WSGI server instead.
 * Debug mode: off
    \end{Verbatim}

    \begin{Verbatim}[commandchars=\\\{\}]
 * Running on http://127.0.0.1:5000/ (Press CTRL+C to quit)
    \end{Verbatim}

    


    
\end{document}
